%% Copernicus Publications Manuscript Preparation Template for LaTeX Submissions
%% ---------------------------------
%% This template should be used for copernicus.cls
%% The class file and some style files are bundled in the Copernicus Latex Package, which can be downloaded from the different journal webpages.
%% For further assistance please contact Copernicus Publications at: production@copernicus.org
%% https://publications.copernicus.org/for_authors/manuscript_preparation.html


%% Please use the following documentclass and journal abbreviations for discussion papers and final revised papers.

%% 2-column papers and discussion papers
\documentclass[journal abbreviation, manuscript]{copernicus}



%% Journal abbreviations (please use the same for discussion papers and final revised papers)


% Advances in Geosciences (adgeo)
% Advances in Radio Science (ars)
% Advances in Science and Research (asr)
% Advances in Statistical Climatology, Meteorology and Oceanography (ascmo)
% Annales Geophysicae (angeo)
% Archives Animal Breeding (aab)
% ASTRA Proceedings (ap)
% Atmospheric Chemistry and Physics (acp)
% Atmospheric Measurement Techniques (amt)
% Biogeosciences (bg)
% Climate of the Past (cp)
% DEUQUA Special Publications (deuquasp)
% Drinking Water Engineering and Science (dwes)
% Earth Surface Dynamics (esurf)
% Earth System Dynamics (esd)
% Earth System Science Data (essd)
% E&G Quaternary Science Journal (egqsj)
% Fossil Record (fr)
% Geochronology (gchron)
% Geographica Helvetica (gh)
% Geoscience Communication (gc)
% Geoscientific Instrumentation, Methods and Data Systems (gi)
% Geoscientific Model Development (gmd)
% History of Geo- and Space Sciences (hgss)
% Hydrology and Earth System Sciences (hess)
% Journal of Micropalaeontology (jm)
% Journal of Sensors and Sensor Systems (jsss)
% Mechanical Sciences (ms)
% Natural Hazards and Earth System Sciences (nhess)
% Nonlinear Processes in Geophysics (npg)
% Ocean Science (os)
% Primate Biology (pb)
% Proceedings of the International Association of Hydrological Sciences (piahs)
% Scientific Drilling (sd)
% SOIL (soil)
% Solid Earth (se)
% The Cryosphere (tc)
% Web Ecology (we)
% Wind Energy Science (wes)


%% \usepackage commands included in the copernicus.cls:
%\usepackage[german, english]{babel}
%\usepackage{tabularx}
%\usepackage{cancel}
%\usepackage{multirow}
%\usepackage{supertabular}
%\usepackage{algorithmic}
%\usepackage{algorithm}
%\usepackage{amsthm}
%\usepackage{float}
%\usepackage{subfig}
%\usepackage{rotating}

\usepackage{todonotes}
\begin{document}

\title{Synergistic radar and sub-millimeter radiometer retrievals
  of ice hydrometeors in mid-latitude frontal cloud systems}

% \Author[affil]{given_name}{surname}

\Author[1]{Simon}{Pfreundschuh}
\Author[2]{Stuart}{Fox}
\Author[1]{Patrick}{Eriksson}
\Author[3]{David}{Duncan}
\Author[4]{Stefan A.}{Buehler}
\Author[4]{Manfred}{Brath}
\Author[2]{Richard}{Cotton}
\Author[5]{Florian}{Ewald}
%\Author[6]{Julien}{Delanoë}

\affil[1]{Department of Space, Earth and Environment, Chalmers University of Technology, 41296 Gothenburg, Sweden}
\affil[2]{Met Office, FitzRoy Road, Exeter, EX1 3PB, United Kingdom}
\affil[3]{ECMWF, Shinfield Park, Reading RG2 9AX, United Kingdom}
\affil[4]{Meteorologisches Institut, Fachbereich Geowissenschaften, Centrum für Erdsystem und Nachhaltigkeitsforschung (CEN), Universität Hamburg, Bundesstraße 55, 20146 Hamburg, Germany}
\affil[5]{Institut für Physik der Atmosphäre, Deutsches Zentrum für Luft- und Raumfahrt, Germany}

\runningtitle{Synergistic radar and sub-millimeter radiometer retrievals
  of ice hydrometeors}
\runningauthor{Simon Pfreundschuh}
\correspondence{Simon Pfreundschuh (simon.pfreundschuh@chalmers.se)}

\received{}
\pubdiscuss{} %% only important for two-stage journals
\revised{}
\accepted{}
\published{}

%% These dates will be inserted by Copernicus Publications during the typesetting process.

\firstpage{1}

\maketitle

\begin{abstract}
  
  Accurate measurements of ice hydrometeors are required to improve the
  representation of clouds and precipitation in weather and climate models. In
  this study, a newly developed, synergistic retrieval algorithm that combines
  radar with passive millimeter and sub-millimeter observations is applied to
  observations of three frontally-generated, mid-latitude cloud systems in order
  to validate the retrieval and asses its capabilities to constrain the
  properties of ice hydrometeors. To account for uncertainty in the assumed
  shapes of ice particles, the retrieval is run multiple times while the 
  shape is varied. Good agreement with in situ measurements of ice water content
  and particle concentrations for particle maximum diameters larger than
  $200\ \unit{\mu m}$ is found for one of the flights for the Large Plate
  Aggregate and the 6-Bullet Rosette shapes. The variational retrieval fits the
  observations well although small systematic deviations are observed for some
  of the sub-millimeter channels pointing towards issues with the sensor calibration or
  the modeling of gas absorption. We find that the quality of the fit to the
  observations is independent of the assumed ice particle shape, indicating that
  the employed combination of observations is insufficient to constrain the
  shape of ice particles in the observed clouds. Compared to a radar-only
  retrieval, the results show an improved sensitivity of the synergistic
  retrieval to the microphysical properties of ice hydrometeors at the base of
  the cloud.

  Our findings indicate that the synergy between active and passive microwave
  observations improve remote-sensing measurements of ice hydrometeors and may
  thus help to reduce uncertainties that affect currently available data
  products. Due to the increased sensitivity to their microphysical properties,
  the retrieval may also be a valuable tool to study ice hydrometeors in field
  campaigns. The good fits obtained to the observations increases confidence in
  the modeling of clouds in the Atmospheric Radiative Transfer Simulator and the
  corresponding single scattering database, which were used to implement the
  retrieval forward model. Our results demonstrate the suitability of these
  tools to produce realistic simulations for upcoming sub-millimeter sensors
  such as the Ice Cloud Image or the Arctic Weather Satellite.

 
\end{abstract}


\introduction  %% \introduction[modified heading if necessary]

The representation of clouds in climate models remains an important issue that
causes significant uncertainties in their predictions \citep{zelinka20}.
Improving and validating these models requires measurements that accurately
characterize the distribution of hydrometeors in the atmosphere. At regional and
global scales, such observations can be obtained efficiently only through remote
sensing. Unfortunately, currently available remote-sensing observations do not
constrain the distribution of ice in the atmosphere well \citep{waliser09,
  eliasson11, duncan18a}.

To address this, the Ice Cloud Imager (ICI) radiometer, to be launched onboard
the second generation of European operational meteorological satellites
(MetOp-SG), will be the first operational sensor to provide global observations
of clouds at microwave frequencies exceeding 183 GHz. Compared to microwave
observations at currently available frequencies ($\leq 183\ \unit{GHz}$),
observations at and above $243\ \unit{GHz}$ have been shown to be sensitive to a
broader size range of hydrometeors \citep{buehler12} as well as their shape and
particle size distribution \citep{evans98}. Although the increased sensitivity
to smaller particles and their microphysical properties is expected to help
improve measurements of ice in the atmosphere, it also increases the complexity
of simulations of cloud observations, which are an essential tool for performing
these measurements in the first place.

In \citet{pfreundschuh20}, we have developed a cloud-ice retrieval based on
radar and passive sub-millimeter observations to investigate the potential
benefits of a synergistic radar mission to fly in constellation with ICI on
MetOp-SG. The simulation-based results from \citet{pfreundschuh20} indicate that
combining active and passive observations across millimeter and sub-millimeter
observations can indeed help to better constrain the distributions of ice
hydrometeors in cloud retrievals. The principal aim of this study is to validate
the synergistic retrieval using real observations and to investigate its ability
to retrieve the vertical distributions of ice hydrometeors.

Since the retrieval has been shown to work on simulated observations, the
validation of the synergistic retrieval essentially amounts to verifying the
physical realism of the underlying forward model. The observations from the
three flights considered here thus also constitute an opportunity to validate
the radiative transfer model that is used in the retrieval, i.e. the Atmospheric
Radiative Transfer Simulator (ARTS, \citeauthor{arts18}, \citeyear{arts18}) and
the corresponding ARTS single scattering database (ARTS SSDB,
\citeauthor{eriksson18}, \citeyear{eriksson18}), to accurately simulate cloud
observations at sub-millimeter wavelengths. Such simulations are of paramount
importance not only for future cloud retrievals from ICI observations
\citep{eriksson20} but also for assimilating cloud-contaminated observations in
numerical weather prediction models \citep{geer17}.


%So far, direct efforts to validate sub-millimeter radiative transfer through
%clouds at frequencies exceeding $183\ \unit{GHz}$ have been limited by the
%availability of such observations and the difficulty of accurately representing
%observed clouds and background atmosphere in the simulations. In the upper
%troposphere, observations of clouds have been obtained as by-products of
%space-borne limb-sounding missions aimed to study gases in the stratosphere.
%Retrievals of ice mass concentrations using sub-millimeter limb-sounding
%observations were developed for AURA MLS \citep{wu06}, Odin/SMR
%\citep{eriksson07} and SMILES \citep{millan13, eriksson14}. However, due to
%their geometry the observations are limited to very high clouds in the tropics.
%Other sources of sub-millimeter observations of clouds are airborne sensors with
%sub-millimeter channels such as the Millimeter-wave imaging radiometer (MIR,
%\citet{wang01}), the Compact Scanning Submillimeter Imaging Radiometer (CoSSIR,
%\citet{evans05}) and the International Sub-Millimeter Airborne Radiometer
%(ISMAR, \citet{fox17}). To the best knowledge of the authors, there are so far
%only two notable efforts to validate microphysical assumptions involved in
%sub-millimeter radiative transfer simulations: Firstly, the study by
%\citet{evans05}, who use a Bayesian retrieval to retrieve integrated radar
%backscatter of a convective anvil and achieve good agreement with simultaneous
%radar measurements at $90\ \unit{GHz}$. And secondly, the study by
%\citet{fox18}, who perform a closure study based on lidar and in situ
%observations of hydrometeors in a mid-latitude cirrus cloud.


In this study, the synergistic retrieval is applied to co-located radar and
microwave radiometer observations of three mid-latitude cloud systems. The
sensitivity of the retrieval to the ice particle shape that is assumed in the
forward simulations is tested by running the retrieval multiple times while
varying the assumed shape. To test the accuracy of the retrieval, retrieval
results are compared to in situ measurements of bulk ice water content (IWC) and
particle size distributions (PSDs) for the two flights where these were
available. Finally, we assess the consistency of the forward model simulations
by investigating the agreement between simulated and real observations as well
as between retrieved atmospheric state and in situ measurements.

The remainder of this article is structured as follows:
Section~\ref{sec:methods_and_data} provides a description of the datasets and
the retrieval algorithm upon which this study is based.
Section~\ref{sec:results} presents the results of the retrieval as well as the
comparisons to in situ data followed by a discussion of those results in
Sec.~\ref{sec:discussion} and conclusions in~\ref{sec:conclusions}.


\section{Data and methods}
\label{sec:methods_and_data}

\begin{figure}[h!]
  \centering \includegraphics[width=1.0\textwidth]{figures/fig01}
  \caption{Flight paths of the cloud overpasses considered in this study. First
    row of panels shows the true-color composite derived from the closest
    overpasses of the MODIS \citep{modis} sensor onboard the Aqua satellite.
    Second row shows ERA5 temperature (colored background),
    geopotential (contours) and wind speed (arrows) at the $800\ \unit{mb}$
    pressure level from \citet{era5}.}
  \label{fig:flight_overview}
\end{figure}

The synergistic retrieval algorithm uses combined observations from radar and
passive microwave sensors to retrieve vertical profiles ice hydrometeor
distributions. The passive observations for this study are taken from the the
MARSS \citep{mcgrath01} and ISMAR \citep{fox17} radiometers on board the UK’s
BAe-146-301 Atmospheric Research Aircraft (FAAM BAe-146) aircraft. Since the
instrumentation of the FAAM BAe-146 aircraft does not include a cloud radar,
only flights for which the radiometer observations can be co-located with radar
observations from another platform are suitable for the combined retrieval.
Since ISMAR is currently the only operational radiometer with channels up to
$664$ and $874\ \unit{GHz}$, the flights considered in this study provide a rare
opportunity to study the synergies between radar and passive (sub-)millimeter
observations using real observations.

An overview of the three flights and the corresponding meteorological contexts
is provided in Fig.~\ref{fig:flight_overview}. The first considered flight, from
14 October 2016 and designated as B984, was part of the North Atlantic Waveguide
and Downstream Impact Experiment (NAWDEX), which took place during September and
October 2016 \citep{schaefler18}. During this flight, a cloud system generated
by an occluded front has been observed quasi-simultaneously by three research
aircraft: The High Altitude and Long Range Research Aircraft (HALO,
\citet{krautstrunk12}), the FAAM BAe-146 and the Falcon 20 of the Service des
Avions Francais Instrumentations pour la Recherche en Environnement (SAFIRE).
The two other flights, designated C159 and C161, were part of the PIKNMIX-F
campaign which took place in March 2019. These two flights were performed
following the ground track of simultaneous overpasses of the CloudSat satellite.
The observations probe clouds in different regions of a frontal system generated
by a low-pressure system passing over Iceland around 21 March 2019. The cloud
system observed during flight C159 is a stratiform, lightly-precipitating cloud
located in the warm sector of the frontal system, whereas the clouds observed
during flight C161 are of convective origin and located in the active region of
the cold front. All datasets that were used in this study are listed together
with their sources in Tab.~\ref{tab:data}.

\begin{table}
  \centering
  \caption{Datasets used in the this study.}
  \begin{tabular}{p{5cm}|p{5cm}|p{3cm}}
     Title &  Usage &  Reference \\
    \hline
    \hline

    HALO Microwave Package measurements during North Atlantic Waveguide and
    Downstream impact EXperiment (NAWDEX)
    & Radar observations for for flight B984
    & \citet{hamp_nawdex} \\
    \hline

    CloudSat 1B-CPR
    & Radar observations for flight C159 (Granule 67658), C161 (Granule 68702)
    &  \citet{tanelli08}  \\
    \hline

    FAAM B984 ISMAR and T-NAWDEX flight: Airborne atmospheric measurements from
    core instrument suite on board the BAE-146 aircraft
    & Radiometer observations and in situ measurements for flight B984
    & \citet{faam_nawdex_obs} \\
    \hline

  FAAM C159 PIKNMIX-F flight: Airborne atmospheric measurements from core and
  non-core instrument suites on board the BAE-146 aircraft
  & Radiometer observations and in situ measurements for flight C159
  & \citet{faam_c159_obs} \\
    \hline

  FAAM C161 PIKNMIX-F flight: Airborne atmospheric measurements from core and
  non-core instrument suites on board the BAE-146 aircraft
    & Radiometer observations for flight C159
    & \citet{faam_c161_obs} \\
    \hline 

    ERA5 global reanalysis
    & A priori state and atmospheric background fields
    & \citet{era5}
  \end{tabular}
  \label{tab:data}
\end{table}


\subsection{Radar observations}

The radar observations from the three flights are displayed in
Fig.~\ref{fig:observations_radar}. Observations for flight B984 stem from the
HAMP MIRA radar \citep{mech14} onboard the HALO aircraft, which operates at a
frequency of $35\ \unit{GHz}$ and has been characterized and calibrated by
\citet{ewald19}. Its observations have been downsampled to a vertical resolution
of $210\ \unit{m}$ and a horizontal resolution of roughly $700\ \unit{m}$ in
order to reduce the computational complexity of the retrieval and to better
match the field of view of the passive observations, which, at an altitude of
$5\ \unit{km}$, vary between about $900\ \unit{m}$ for the low- frequency
channels and $200\ \unit{m}$ for the high-frequency channels.

The radar observations for flights C159 and C161 stem from the CloudSat Cloud
Profiling Radar (CPR, \citet{tanelli08}), which operates at $94\ \unit{GHz}$.
Since the CloudSat observations were affected strongly by ground-clutter, the
first five bins located completely above surface altitude were set to the
reflectivity found in the sixth bin above the surface. The CPR observations
have a vertical resolution of $240\ \unit{m}$ and a horizontal resolution of
about $1.4\ \unit{km}$. The horizontal distance between subsequent observations is
$1.1\ \unit{km}$.

\begin{figure}[hbpt!]
  \centering
  \includegraphics[width=0.8\textwidth]{figures/fig02}
  \caption{Radar observations from the flights used in this study. Panel (a)
    shows the radar reflectivity measured by the HAMP MIRA $35\ \unit{GHz}$
    cloud radar. Panels (b) and (c) show the reflectivity measured by the
    CloudSat CPR at $94\ \unit{GHz}$. The white line displays the ERA5 freezing
    level from \citet{era5}. }
  \label{fig:observations_radar}
\end{figure}

\subsubsection{MARSS}

The MARSS radiometer measures microwave radiances at $89\ \unit{GHz},
157\ \unit{GHz}$ and channels located around the water vapor line at $183
\ \unit{GHz}$. Although MARSS is a scanning radiometer only observations within
$5\ \unit{^\circ}$ off nadir are used in the retrieval. The observations from
the three flights are displayed in Fig.~\ref{fig:observations_marss}.
Observations from channels that are sensitive to surface emission
($89\ \unit{GHz}$ and $157\ \unit{GHz}$) are excluded from the retrieval for
flight sections over land. The MARSS observations were mapped to the radar
observations using nearest-neighbor interpolation.

\begin{figure}[h!]
  \centering
  \includegraphics[width=1.0\textwidth]{figures/fig03}
  \caption{
    Passive microwave measurements from the MARSS radiometer together with the
    matched radar observations. Grey background in the radiance plots marks
    observations that were taken over land.
    }
  \label{fig:observations_marss}
\end{figure}


\subsubsection{ISMAR}

\begin{figure}[h!]
  \centering
  \includegraphics[width=1.0\textwidth]{figures/fig04}
  \caption{
    Passive microwave measurements from the ISMAR radiometer together with the
    matched radar observations. Grey background in the radiance plots marks
    observations that were taken over land and are therefore not used in
    the retrieval.
    }
  \label{fig:observations_ismar}
\end{figure}

The ISMAR radiometer has channels covering the frequency range from
$118\ \unit{GHz}$ up to $874\ \unit{GHz}$. As for MARSS, only observations
within $5\unit{^\circ}$ degrees off nadir are used in the retrieval. The
observations from the 3 flights are displayed in
Fig.~\ref{fig:observations_ismar}. Similar as for the two low-frequency
channels of MARSS, the 4 outermost channels around the $118\ \unit{GHz}$ oxygen
line are not used over land. The matching of ISMAR observations to radar
observations is performed in the same way as for MARSS. It should be noted that
not all channels were available on all flights: The channels around
$448\ \unit{GHz}$ were not available on the B984 flight, while the two of the
channels around $325\ \unit{GHz}$ were missing for the C159 and C161 flights.
From the channels at $874\ \unit{GHz}$ only the V polarization was available
for flights C159 and C161.

The polarized measurements at $243\ \unit{GHz}$ and at $664\ \unit{GHz}$ for
flight B984 were replaced by the average of the measured H and V polarizations.
For flights C159 and C161, only the horizontally-polarized measurements at
$664\ \unit{GHz}$ were used used due to excessive noise in the V channel.

\subsection{In situ measurements}
\label{sec:in_situ}

The in situ measurements that are relevant to this study are bulk ice water
content measured using a Nevzorov hot-wire probe \citep{korolev13} and PSDs
recorded using DMT CIP-15 and CIP-100 probes, which measure size-resolved
particle concentrations with resolutions of 15 and $100\ \unit{\mu m}$,
respectively.

In situ measurements of cloud hydrometeors were performed during the flights
B984 and C159. These flights consist of two parts: A high-level run during which
the aircraft flew above the cloud system and during which the remote sensing
observations were performed and a low-level run during which the aircraft
descended through the cloud to perform the in situ measurements. A detailed view
of the high- and low-level runs for the two flights are provided in
Fig.~\ref{fig:flight_overview_detail}. For flight C159, this view reveals a
noticeable horizontal offset of $3$ to $4\ \unit{km}$ between the ground tracks
of radar and radiometer observations. Even larger deviations occur between
certain parts of the low level run and the ground tracks of the remote sensing
observations.

\begin{figure}[h!]
  \centering
  \includegraphics[width=1.0\textwidth]{figures/fig05}
  \caption{
    Detailed view of the flight paths of the high-level runs and in situ sampling
    paths for flights B984 and C159. The backound is the true-color composite
    derived from the closest overpasses of the MODIS \citep{modis} sensor
    on the Aqua satellite.
    }
  \label{fig:flight_overview_detail}
\end{figure}

An overview of the in situ measured IWC and PSDs is provided in Fig.~\ref{fig:in_situ}.
While for flight B984 the measured IWC are mostly consistent with the radar
observations, there are clear disparities between the measured IWC and the CPR
reflectivities for flight C159. This indicates that there may be considerable
differences between the regions of the cloud that were sampled during the
in situ sampling and the part the was observed by the CloudSat CPR.

The PSD profiles for flight B984 show a clear size-sorting pattern with a
gradual decrease of the concentration of particles smaller than $200\ \unit{\mu
  m}$ and a simultaneous increase of the concentration of larger particles. For
flight C159, high concentrations of small particles are encountered at low
altitudes which decrease with altitude. For larger particles no systematic
variation with altitude is observed.

\begin{figure}[hbpt!]
  \centering
  \includegraphics[width=1.0\textwidth]{figures/fig06}
  \caption{ in situ measured IWC and PSDs for flights B984 and C159. The first
    row of panels displays the measured IWC along the flight path plotted on top
    of the co-located radar observations. The second row displays the variation
    of the mean of the in situ measured PSDs for different altitudes in the
    cloud.
    }
  \label{fig:in_situ}
\end{figure}

\subsection{Retrieval algorithm}
\label{sec:synergistic_retrieval}

The synergistic retrieval algorithm used in this study is based on the optimal
estimation framework \citep{rodgers00} and retrieves distributions of frozen and
liquid hydrometeors together with water vapor by simultaneously fitting a
forward model to the active and passive observations. Since the algorithm is
described in detail in \citet{pfreundschuh20} the following section only
outlines its main features and how it has been adapted to the flight data.

The retrieval input consists of a single radar profile and the corresponding
spatially closest radiometer observations. Background properties of atmosphere
and the surface, such as temperature and wind speed, as well as a priori profiles
for relative humidity and liquid cloud water are taken from the ERA5 hourly
reanalysis \citep{era5}. The output of the retrieval are two parameters of the
PSDs of frozen and liquid hydrometeors as well as liquid cloud water content
(LCWC) and relative humidity. Concentrations of hydrometeors are represented
using the normalized PSD approach proposed by \citet{delanoe05}. At each level
in the atmospheric column the concentration of hydrometeors with respect to the
volume equivalent $D_\text{eq}$ is given by:
\begin{align}
  N(D_\text{eq}) &= N_0^{*}F(\frac{D_\text{eq}}{D_m})
\end{align}
Here $F$ is a fixed function that specifies the shape of the normalized PSD. The
parameters $N_0^{*}$ and $D_m$ are the free parameters that are retrieved
by the retrieval. The parameters are retrieved for one liquid and one frozen species
of frozen hydrometeors, with $N_0^{*}$ retrieved in log-space and $D_m$ in linear
space. In addition to that, the retrieval retrieves relative humidity transformed
using an $\text{arctanh}$ transform and cloud liquid water content also transformed
using a $\text{log}$ transform. All retrieval targets and corresponding a priori
assumptions are listed in Tab.~\ref{tab:a_priori}.

\begin{table}[h!]
  \caption{Retrieval quantities and a priori assumptions used in the retrieval.
    The relation for the a priori mean of $\log_{10}(N_0^*)$ is taken from
    \citep{cazenave19}.}
 \centering
\label{tab:a_priori}
    \begin{tabular}{c|l|p{5cm}|r}
     Quantity & Retrieved parameters & A priori mean &  A priori std. dev. \\
    \hline
    \hline
    Ice water content (IWC)
    & $\log_{10}(N_0^*)$
    & $-0.076586 \cdot (T - 273.15) + 17.948$ with $T$ temperature in $\unit{K}$
    & 2 \\
    & $D_m$
    & $\text{IWC} = 10^{-6}$
    & 500\ \unit{\mu m} \\
    
    \hline
    Rain water content (RWC)
    & $\log_{10}(N_0^*)$
    & 7 & 2 \\
    & $D_m$
    &  500\ \unit{\mu m}
    & 500\ \unit{\mu m}  \\
    \hline
    Cloud liquid water content (CLWC)
    & $\log_{10}(\text{CLWC})$
    & From ERA5
    & 1 \\

    \hline
    Relative humidity (RH) &
    $\text{arctanh}(\frac{2 \cdot \text{RH}}{1.1} - 1.0)$
    & From ERA5
    & 1 \\
    \end{tabular}
\end{table}

The forward model and retrieval were made adaptive so that the ingested
observations can be easily adapted to the different sensors and channels that
were available for each flight. Low frequency channels that are used only over
Ocean surfaces are deactivated by setting the corresponding channel uncertainty
to $10^6\ \unit{K}$. The atmospheric grid was limited to altitudes between $0$
and $10\ \unit{km}$ and matched to the resolution of the radar observations. The
latest stable release (version 2.4) of ARTS \citep{buehler18} is used to implement the forward model used in
the retrieval. The built-in single-scattering radar solver of ARTS is used to
calculate radar observations and Jacobians. To account for the effect of
multiple scattering in CloudSat observations, the attenuation due to
hydrometeors is scaled at each atmospheric layer by a factor of $0.5$ following
Fig.~16 in \citet{battaglia10}. Passive radiances are calculated using the ARTS
interface to DISORT \citep{disort00} and their Jacobians are approximated using
a first order scattering approximation. Gaseous absorption is modeled using the
absorption models from \cite{rosenkranz93} for $N_2$ and $O_2$. Following
\citet{fox20}, absorption from water varpor is calculated using a combination of
the AER database v3.6 \citep{aer36} for resonant absorption and the MT-CKD model
version 3.2 for continuum absorption \citep{mlawer12}.

\subsubsection{Representation of frozen hydrometeors}

The forward model simulates active and passive observations in two steps: In the
first one, the bulk properties that are used to represent hydrometeors in the
retrieval are mapped to corresponding optical properties. The optical properties
are then, in the second step, used together with background atmosphere and
surface to simulate the observations.

The mapping of bulk to optical properties is based on a PSD and an ice particle
habit that associates particles of different sizes and shapes to optical
properties. More specifically, the forward model uses the normalized PSD
approach proposed by \citet{delanoe05} with the mass-weighted mean diameter
($D_m$) and intercept parameter ($N_0^*$) as parameters. The updated values from
\citet{cazenave19} are used as shape parameters of the distribution. The ice
particle habit is represented by a collection of ice particle shapes and
corresponding, pre-computed single particle optical properties. Bulk optical
properties are calculated by integrating the product of particle density and
optical properties over the particle size. As the retrieval is currently set up,
the particle habit cannot be retrieved and must be assumed a priori. Due to the
large variability of ice particle shapes in real clouds, it is unclear which
particle habit should be chosen to best represent their radiative properties or
whether such a unique best model exists at all. The approach taken here is
therefore to select a set of habits and perform the retrieval with each of them.
This will allow us to investigate the impact of the selected habit on the
results.

Five particles were selected from the set of standard habits that is distributed
with the ARTS SSDB \citep{eriksson18}. The standard habits are particle mixes
that combine pristine crystals at small sizes with aggregate shapes at larger
sizes. The selected habits are listed in Tab.~\ref{tab:particle_habits}. To
provide an overview of their optical properties, characteristic bulk optical
properties have been calculated and displayed in Fig.~\ref{fig:particle_properties}
together with their mass-size relationships. The PSD used to calculate the bulk
optical properties is the same that is used in the retrieval with the $N_0^*$
value set to the a priori value at a temperature of $260\ \unit{K}$. The
particles were selected so that their properties cover most of the variability
of the available set of standard habits both in terms of the mass-size
relationship as well as their optical properties.

\begin{figure}[hbpt!]
  \centering
  \includegraphics[width=0.9\textwidth]{figures/fig07}
  \caption{Properties of the selected ice particle shapes that are used to
    represent frozen hydrometeors in the retrieval forward model. Colored lines
    display the properties of the selected habits, while grey lines show the
    properties of the remaining standard habits distributed with the ARTS SSDB.
    Bulk optical properties were calculated using the PSD parametrization that
    is used in the retrieval. }
  \label{fig:particle_properties}
\end{figure}

\begin{table}
  \centering
  \caption{Particle habits used in the retrieval. The mass size relationship is given
    in terms of the parameters
    of a fitted power law of the form $m = \alpha \cdot D_\text{MAX}^\beta$ with
    $D_\text{MAX}$ the maximum diameter and $m$ in $\unit{kg}$.}
  \begin{tabular}{l|l|rr|rr}
    \multicolumn{1}{c|}{Habit name} & \multicolumn{1}{c|}{Shapes used} &
    \multicolumn{2}{c|}{Size range} & \multicolumn{2}{c}{Mass size relationship}
    \\
    & Name (ID) &$D_{\text{eq}, \text{ min}}\ [\unit{\mu m}$] &
    $D_{\text{eq}, \text{ max}}\ [\unit{\mu m}]$ &\hfill
    $\alpha$ & \hfill $\beta$ \\
    \hline 
    %
    % 6-Bullet Rosette
    %
    \hline 6-Bullet Rosette & 6-Bullet Rosette (6) & $16\ $ & $8905\ $ & \hfill 0.4927 & \hfill 2.4278 \\

    %
    % Large Plate Aggregate
    %
    \hline 8-Column Aggregate & 8-Column Aggregate (8) & $10\ $ & $3000\ $ & \hfill 440 & \hfill 3 \\

    %
    % Evans Snow Aggregate
    %
    \hline Evans Snow Aggregate & Evans Snow Aggregate (1) & $50\ $ & $2109\ $ & \hfill 0.196 & \hfill 2.386 \\

    %
    % Large Plate Aggregate
    %
    \hline Large Plate Aggregate & Thick Plate (15) & $16\ $ & $200\ $ & \hfill
    110 & \hfill 3 \\ & Large Plate Aggregate (20) & $160\ $ & $3021\ $ & \hfill
    0.21 & \hfill 2.26 \\
    %
    % Large Column Aggregate
    %
    \hline Large Column Aggregate & Block Column (12) & $10\ $ & $200\ $ &
    \hfill 110 & \hfill 3 \\ & Large Column Aggregate (18) & $160\ $ & $3021\ $
    & \hfill 0.25 & \hfill 2.43 \\

  \end{tabular}
  \label{tab:particle_habits}
\end{table}

Complementary information that can help guide the selection of a suitable
particle shape can be obtained from in situ measurements. Since the particle
habit associates particle sizes with a specific shape it can be used to compute
a bulk water content corresponding to PSD measurements. This allows calculating
the IWC corresponding to the in situ measured PSDs, which can be compared with
the IWC measured by the Nevzorov probe. The agreement between the PSD-derived
IWC and the in situ measured IWC can then provide insight into whether the
mass-size relation corresponding to the particle shape is consistent with
that of the particles in the cloud. Such a comparison is provided in
Fig.~\ref{fig:mass_size_relation}.

For both flights, the Large Plate Aggregate and the 6-Bullet Rosette yield
the best overall agreement with the in situ measured IWC. The Large Column
Aggregate yields values at the low end of the measured distribution for all
flights and altitudes. The Evans Snow Aggregate yields similar results to those
of the Large Column Aggregate except at high altitudes for flight B984 and low
altitudes for flight C159. The 8-Column Aggregate generally yields higher IWC
values than most other habits and tends to overestimate the in situ IWC at low
altitudes for flight B984.

\begin{figure}
  \centering
  \includegraphics[width=1.0\textwidth]{figures/fig08}
  \caption{ Comparison of bulk IWC as measured by the Nevzorov probe and
    inferred from in situ measured PSDs using a given particle shape. The
    background in each plot shows the distribution of Nevzorov-measured IWC for
    a given $1\ \unit{km}$-altitude bin. Colored boxes display the distribution
    of IWC in that bin inferred for a given particle shape. Boxes, whiskers and
    outliers are drawn following Tukey's conventions for box plots.}
  \label{fig:mass_size_relation}
\end{figure}


\section{Results}
\label{sec:results}

The primary results of the combined retrieval are the retrieved hydrometeor
size distributions. In addition to that, the retrieval also fits a radiative transfer
model to the observations whose agreement with the real observations can provide
valuable information regarding the accuracy of the forward model and the fitness
of a priori and modeling assumptions.

\subsection{Fit to observations}

The retrieval residuals, i.e. the difference between simulated and real
observations, for the Large Plate Aggregate habit are displayed in
Fig.~\ref{fig:residuals}. As the figure shows, the retrieval was able to fit
both radar and radiometer observations fairly well for all flights. For flight
B984, the radar residuals show some scattered deviations located at the edge of
the cloud, which are likely discretization artifacts. Except for that, residuals
for this flight remain well within $1\ \unit{dB}$. The residuals for flight
C159, exhibit four vertical stripes with significant residuals in the radar
observations. In these regions, which correspond to significant scattering
depressions in most passive channels up to $325\ \unit{GHz}$, the simulations
overestimate the radar reflectivity. Apart from this, there are some smaller
regions where the simulations underestimate the radar reflectivity but these
remain limited to within few $\unit{dB}$. For flight C161, moderate negative
residuals in the radar observations can be observed in the right half of the
convective core, which coincide with an overestimation of the scattering signal
at $243\ \unit{GHz}$.

Radiometer residuals for flight B984 are mostly within $\pm 5\ \unit{K}$. For
the two other flights the residuals are larger. For the other two flight,
differences exceeding $10\ \unit{K}$ are observed in the window channels up to
$243\ \unit{GHz}$ as well as the outermost channels around the absorption lines
at $118\ \unit{GHz}$ and $183\ \unit{GHz}$. Since these correspond to profiles
in which similar residuals occur in the radar observations, a likely explanation
is that they are caused by liquid precipitation that is not observed by all sensors
due to co-location issues.


\begin{figure}[!hbpt]
  \centering
  \includegraphics[width=0.95\textwidth]{figures/fig09}
  \caption{Differences between observed and fitted observations for the Large
    Plate Aggregate particle. First two rows depict the radar observations and
    their residuals, respectively. Following rows show the retrieval residual in
    the radiometer measurements for each of the frequency bands used in the
    retrieval. The grey shading marks sections of the flight path that were
    located over land surfaces. }
  \label{fig:residuals}
\end{figure}

For a more systematic analysis of the effect of the assumed particle shape on
the retrieval residuals, their distribution for radar and radiometer channels
around $183\ \unit{GHz}$ and above are displayed in
Fig.~\ref{fig:residuals_box}. The distributions, which for most channels are
close to or centered around zero, confirm that the retrieval generally fits the
observations well. The largest deviations are observed for the $874\ \unit{GHz}$
channel and the $243\ \unit{GHz}$ channel for flight C161. For flights C159 and
C161, the $874\ \unit{GHz}$ and $664\ \unit{GHz}$ channels exhibit small
systematic biases of opposite signs, which may indicate issues with the
calibration or the modeling of water vapor absorption at these channels.
Furthermore, it is interesting to note that the ice particle habit only has a
minor impact on the residuals indicating that the retrieval can compensate for
mismatches in the assumed particle shape by adjusting the retrieval variables.


\begin{figure}[!hbpt]
  \centering
  \includegraphics[width=0.8\textwidth]{figures/fig10}
  \caption{Distributions of retrieval residuals for different particle
    shapes used in the forward model for each of the three flights.}
  \label{fig:residuals_box}
\end{figure}

\subsection{Retrieved ice water content}

The retrieved bulk IWC and corresponding IWP for all three cloud scenes are
displayed in Fig.~\ref{fig:ice_water_content}. For all three flights, the ice
particle shape has a significant effect on the retrieved amount of ice. In terms
of IWP, the Large-Column Aggregate and Evans Snow Aggregate habits yield the
highest values, while the 8-Column Aggregate consistently yields the lowest IWP.
The Large Plate Aggregate and 6-Bullet Rosette both yield values within the
range of the other particle models with the 6-Bullet Rosette leading to slightly
higher IWP values than the Large Plate Aggregate. In addition to the effect of
the increased total retrieved water content, the particle habit also has a small
effect on the vertical distribution of the ice hydrometeors, which is visible
particularly for retrieved IWC in the convective core observed during flight
C161.


\begin{figure}[!hbpt]
  \centering
  \includegraphics[width=1.0\textwidth]{figures/fig11}
  \caption{Retrieved IWP and IWC content for all flights and different ice
    particle shapes assumed in the retrieval. The white line displays the ERA5
    freezing level from \citet{era5}.
  }
  \label{fig:ice_water_content}
\end{figure}

\subsection{Comparison to in situ measurements}

The most important question regarding the hydrometeor retrieval is certainly
whether the retrieved bulk properties are consistent with the in situ
measurements. To compare the in situ measurements to the retrieval results, they
were mapped to the radar observations using a nearest-neighbor criterion. For
B984, retrieval results within a distance of 1km of the flight path were then
associated to the in situ measurements. Because of the mismatch between
observations and in situ sampling paths, another approach was taken for flight
C159. Here the retrieval results were mapped to the in situ measurements by
selecting all retrieval results between $50$ and $150\ \unit{km}$ along-track
distance. Both, the matched retrieval results and the vertically-resolved
distributions of measured and retrieved IWC are displayed in
Fig.~\ref{fig:in_situ_iwc}.

For flight B984, the distribution of in situ measured IWC values is well within
the range of retrieved IWC values across all particle shapes up to an altitude
of around $7\ \unit{km}$. At these altitudes, the best match to the in situ
measurements is achieved with the 6-Bullet Rosette particle and the Large Plate
Aggregate. The 8-column aggregate underestimates the in situ-measured IWC while
Large Column Aggregate and Evans Snow Aggregate overestimate it. Above
$7\ \unit{km}$ all particles lead to results that underestimate the
in situ measured IWC. A likely cause for this is the high concentration of
small particles as observed in the in situ measurements (c.f.
Fig.~\ref{fig:in_situ}) for which microwave observations lack
sensitivity.

For flight C159, the distribution of retrieved IWC still covers the distribution
of in situ measured values for altitudes above $3\ \unit{km}$ but exhibits a
tendency towards underestimation. Overall, the differences between the results
for different habits are smaller for this flight. However, the uncertainties
caused by the large sampling region as well as the potential co-location issues
affecting the results make them less conclusive.

\begin{figure}[!hbpt]
  \centering
  \includegraphics[width=1.0\textwidth]{figures/fig12}
  \caption{in situ measured and retrieved IWC for flights B984 and C159. The
    first row of panels shows the in situ sampling paths in relation to the
    measured radar reflectivity as well as the retrieval values that are mapped
    to the in situ measurements (blue shading). The second row of panels shows
    in the background the distribution of in situ measured IWC values in
    altitude bins with a height of $500\ \unit{m}$. Colored boxes display the
    corresponding distribution of retrieved IWC values for different ice
    particle shapes. Boxes are drawn following Tukey's conventions for box
    plots. The colored triangles mark the mean of the distribution.}
  \label{fig:in_situ_iwc}
\end{figure}

Finally, we want to address the question whether the representation of cloud
microphysics within the retrieval forward model is consistent with the in situ
measured PSDs. For this, we calculate the PSDs corresponding to the retrieved
bulk properties and compare them to the in situ measurements. The results of the
comparison are displayed in Fig.~\ref{fig:in_situ_psds}. For flight B984, we
find good agreement between retrieved and in situ measured PSDs for larger
particles ($D_\text{MAX} > 200 \ \unit{\mu m}$) for the Large Plate Aggregate
and the 6-Bullet Rosette. Since this is observed even at altitudes above $6
\ \unit{km}$, it confirms that the underestimation of IWC at these altitudes is
likely caused by the high concentration of smaller ice particles. For flight
C159, the retrieved PSDs deviate significantly from the in situ measurements.
Although the 6-Bullet Rosette and Large Plate Aggregate seem to fit the tail
($D_\text{MAX} > 1 \unit{mm}$) of the PSD for most altitudes except between $3 -
4\ \unit{km}$, the measured PSDs deviate considerably at smaller sizes. This may
also indicate that the the assumed shape of the PSD may not be suitable for the
observed cloud system.

\begin{figure}[!hbpt]
  \centering
  \includegraphics[width=1.0\textwidth]{figures/fig13}
  \caption{in situ measured and retrieved PSDs for flights B984 (left column)
    and C159 (right column). Each row of panels shows the mean of the
    in situ measured PSDs (black) together with randomly drawn samples of
    measured PSDs (light grey) for a given altitude bin of a height of one
    kilometer. Colored lines on top show the corresponding mean retrieved PSD
    for different assumed particle shapes.}
  \label{fig:in_situ_psds}
\end{figure}

\section{Discussion}
\label{sec:discussion}

This study used a novel, synergistic retrieval to retrieve vertically-resolved
distributions of ice hydrometeors from co-located radar and microwave radiometer
observations. For most of the considered channels, the retrieval succeeded in
fitting both the active and passive observations without significant, systematic
deviations. For two of the flights, the retrieved hydrometeor distributions were
compared to in situ measurements. For one of the flights (B984), we found good
agreement with the in situ measured bulk IWC for altitudes between $2$ and
$6\ \unit{km}$ for two of the particle shapes. The same particle shapes also
yield the best agreements with the in situ measured PSDs for larger ice
particles ($D_\text{MAX} > 200\ \unit{\mu m}$). For the second flight (C159), no
consistency was found between the in situ-measurements and the retrieval. A
likely explanation for this is that the in situ measurements are not as well
co-located due to the temporal and spatial differences between the different
observations as well as the in situ measurements.

\subsection{Sub-millimeter radiative transfer in cloudy atmospheres}

 A first important result of this study is the ability of the retrieval to find
 atmospheric states that are consistent with the observed radiances and radar
 reflectivities for all three flights. This in itself is not self-evident due to
 the uncertainties that still affect the modeling of ice-particle scattering at
 millimeter and sub-millimeter wavelengths. Previous studies that tried to
 directly validate sub-millimeter RT through clouds were either limited to
 tropical clouds \citep{evans05, eriksson07} or cirrus clouds \citep{fox17}. For
 flight B984, the radar and all passive observations were fitted up to small
 systematic biases no larger than $3\ \unit{K}$. The deviations for the two
 other flights were generally larger, but these were likely caused by spatial
 and temporal co-location errors. This indicates that both the assumed optical
 properties as well as the retrieval forward model are consistent across the
 considered wavelengths. Furthermore, the two particle shapes for which the best
 agreement between retrieved and in situ measured hydrometeor distributions was
 found for flight B984, were also those whose mass-size relationship yielded the
 best agreement between in situ measurements of IWC and PSDs. Since this ties
 the microphysical properties of the particles to their optical properties, it
 suggests that the modeling of these particles in the ARTS SSDB is physically
 consistent.

\subsection{The impact of assumed ice particle shape}

A rather unexpected result that emerged from this study is that the retrieval
can fit the observations regardless of the assumed ice particle shape. This
indicates that although the observations are sensitive to variations in ice
particle shape, they alone cannot constrain it. This is in agreement with what
has been reported in \citet{pfreundschuh20}, namely that no correlation could be
found between the particle shape yielding best retrieval fit and the one
yielding the most accurate retrieval results.

In an effort to better separate a potential signal from the ice particle shape
in the retrieval residuals, Fig.~\ref{fig:residuals_scatter} in the appendix
displays the relation between IWP and corresponding residuals in the $325 \pm
3.5 \unit{GHz}$ channel. This channel was chosen because it belongs to the
channels displaying the largest differences between residual distributions for
different particle habits (Fig.~\ref{fig:residuals_box}). Nonetheless, the
plots exhibit no sign of a relationship between either residual and ice
water content or the residuals across different habits.

This result implies that future ice hydrometeor retrievals that make use of
millimeter and sub-millimeter microwave observations must either account for the
uncertainty caused by variations in ice particle shape or find ways to more
accurately constrain the shape a priori. Moreover, for studies that seek to
validate model predictions by comparing simulated and observed microwave
observations, this implies that care must be taken to accurately characterize
the ice particle shape. This is because consistency between simulations and
observations can be achieved for bulk water contents that vary by almost an order of
magnitude (c.f. Fig.~\ref{fig:ice_water_content}).

\subsection{Representation of cloud microphysics}

The lack of a signal that constrains the ice particle shape even in the combined
observations puts additional weight on the question of how to best represent ice
particles in simulations of microwave observations. The habits that lead to the
most accurate retrieval results in this study were the Large Plate Aggregate and
the 6-Bullet Rosette. It is interesting to note that the Large Plate Aggregate
was also found to yield the best agreement between NWP-model-based simulations
and satellite observations at frequencies between $19$ and $190\ \unit{GHz}$ for
stratiform snow in \citet{geer21}.

Nonetheless, these findings are based on observations from the single flight for
which the retrieval results could be reliably compared with in situ
measurements. This result can thus be seen as indication that these habits may
work well for similar mid-latitude cloud systems but more generally applicable
conclusions would require further and more systematic investigation.

\subsection{Retrieval validation}

Since the results presented in \citet{pfreundschuh20} were limited to
simulations based on a high-resolution climate model, the validation of the
retrieval using real observations remained an open issue. For flight B984 good
agreement was found between retrieval results and in situ measurements. Although
the retrieved IWC deviates from the in situ measurements at altitudes $>
6\ \unit{km}$, the retrieved PSDs still match the in situ measurements well for
particles with $D_\text{MAX} > 200\ \unit{\mu m}$. This indicates that the
concentrations of ice particles that the sensors are sensitive to are retrieved
correctly but that the total IWC is still too low because of the underestimation
of smaller particles that is caused by the mismatch between the assumed and
actual PSD shape. Although \citet{oshea21} and \citet{oshea19} show that the
occurrence of high particle concentrations at sizes below $200\ \unit{\mu m}$
may be due to measurement inaccuracies of the CIP-15 probe, the measured PSDs
correctly reproduce the measured IWC at these altitudes when the corresponding
water content is calculated using any of the tested particle habits
(Fig.~\ref{fig:mass_size_relation}). This means that the PSD measurements are
consistent with the IWC measurements across a wide range of realistic mass-size
relationships. Furthermore, the presence of a cloud layer with a large number of
small particles was also reported by \citet{ewald21} who investigated the same
cloud system with combined radar-lidar observations.

For flight C159 no good agreement was found between retrieved and in situ
measured IWC and PSDs. However, some evidence suggest that this may be due to
co-location: Firstly, the flight path for the in situ sampling was found to be
offset from the high-level run during which the observations were taken in the
direction opposite to the wind at $800\ \unit{mb}$
(Fig.~\ref{fig:flight_overview}, Fig.~\ref{fig:flight_overview_detail}).
Secondly, a clear backscattering signal is present in the CloudSat CPR
observations even in regions where only negligible amounts of IWC are present in
the in situ measurements (Fig.~\ref{fig:in_situ_iwc}). Thirdly, the residuals
observed in Fig.~\ref{fig:residuals} are indicative of additional co-location
issues between radar and radiometer observations. Finally, also the
comparison of retrieved and in situ measured PSDs (Fig.~\ref{fig:in_situ_psds})
seems to indicate large deviations between the observed and the assumed PSD
shape.

\subsection{The added value synergistic cloud retrievals}

Although the evidence from flight B984 suggests that the synergistic
retrieval algorithm works well for retrieving ice hydrometeor concentrations,
similar retrievals can be performed using only radar observations. A retrieval
using only radar observations has the obvious advantage of requiring only
a single sensor and being computationally much less complex. This naturally
leads to the question of the added value that a synergistic retrieval
can provide.

To investigate this, the results of the combined and an equivalent radar-only
retrieval for flight B984 are displayed in
Fig.~\ref{fig:in_situ_iwc_radar_only}. For the Large Plate Aggregate and
6-Bullet-Rosette habits, the results of the combined and the radar-only
retrieval are largely similar down to an altitude of about $3.5\ \unit{km}$
below which the radar-only retrieval tends to overestimate the in situ IWC. In
contrast to the combined retrieval, the results of the radar-only retrieval
exhibit almost no impact from the particle habit. So while the radar-only
retrieval remains mostly unaffected by the habit choice, using a different habit
in the combined retrieval may cause systematic overestimation (Evans Snow
Aggregate and Large Column Aggregate) or underestimation (8-Column Aggregate).

\begin{figure}[!hbpt]
  \centering
  \includegraphics[width=1.0\textwidth]{figures/fig14}
  \caption{in situ measured and retrieved PSDs for flight B984
    for the combined and the radar-only retrieval. Each panel displays
    the distributions of in situ measured IWC for different altitude
    bins in the background and the distribution of retrieved IWC values
    for different ice habits as colored boxes on top.}
  \label{fig:in_situ_iwc_radar_only}
\end{figure}

A similar comparison is shown in Fig.~\ref{fig:in_situ_psds_radar_only} for the
retrieved PSDs. The PSDs are largely similar for both retrievals for altitudes
above $4\ \unit{km}$. Below that, however, the radar-only retrieval
overestimates the particle concentrations, while the combined retrieval matches
the in situ measurements well for the Large Plate Aggregate and 6-Bullet Rosette
habits. This indicates that the combined retrieval utilizes the complementary
information in the radar and passive observations to match both moments of the
PSD, whereas the radar-only retrieval can only match one of them.

\begin{figure}[!hbpt]
  \centering
  \includegraphics[width=1.0\textwidth]{figures/fig15}
  \caption{in situ measured and retrieved PSDs for flight B984
    retrieved using the combined (panel (a)) and the radar-only retrieval
    (panel (b)). Each row of panels shows the mean of the in situ measured
    PSDs (black) together with randomly drawn samples of measured PSDs
    (light grey) for a given altitude bin of a height of one kilometer.
    Colored lines on top show the corresponding mean retrieved PSD for
    different assumed particle shapes.}
  \label{fig:in_situ_psds_radar_only}
\end{figure}

These results thus suggest that combining radar with passive microwave
observations helps to constrain the PSD of ice hydrometeors for sufficiently
large particle sizes ($D_\text{MAX} > 200\ \unit{\mu m}$). Since for air- and
space-borne observations only microwave observations can sense the base of thick
clouds, this is a unique synergy between these types of observations.

\subsection{Limitations}

Since microwave radiative transfer simulations in cloudy atmospheres remain a
challenging problem, it is important to also consider the limitations of the
simulations and derived results that were presented in this study. The
simplifications that were applied in the simulations are the following:
\begin{enumerate}
  \item Horizontal photon transport between the retrieved profiles is ignored.
  \item Inhomogeneity across the radar and radiometer beams is ignored.
  \item The finite spectral resolution of the passive channels is neglected.
  \item The radar solver neglects multiple scattering.
  \item The effects of particle orientation are ignored.
\end{enumerate}

\citet{barlakas20} found that neglecting photon transport in simulations of
sub-millimeter observations across a footprint of $6\ \unit{km}$ incurs only a
small random error with biases $< 0.5\ \unit{K}$, so it is likely also small for
the simulations presented here. For flight B984, the horizontal averaging of the
radar observations leads to a profile width of $700\ \unit{m}$ which is fairly
close to the width of the radiometer field of views which varies between about
$900$ and $200\ \unit{m}$ at an altitude of $5\ \unit{km}$. The effect of beam
inhomogeneity is therefore expected to be small for this flight. For flights
C159 and C161, the radar beam has an along-track width of about
$1.4\ \unit{km}$, which is larger than that of the radiometers, so these
observations may be affected to a larger extent than those for flight B984.

Neglecting the finite spectral resolution of the passive channels can lead to an
error of up to $2.1\ \unit{K}$ for satellite observations that are affected by
Ozone absorption \citep{eriksson20}. Since the passive observations used in this
study were all taken from altitudes below $10\ \unit{km}$ the effect of this
approximation is likely negligible.

The effect of multiple scattering for air-borne radar observations is generally
negligible \citep{battaglia10}. For CloudSat observations, however, the higher
frequency and the considerably wider footprint will increase the effects of
multiple scattering on the observations. Although the simulations account for
the signal-enhancing effect of multiple scattering by layer-wise reduction of
the attenuation, the presence of multiple scattering may still add to the
uncertainty in the simulations for flights C159 and C161.

Finally, there is the potential presence of oriented particles in the cloud. The
different ice habits used in this study all assume totally random orientation of
the ice particles. Systematic vertical orientation of particles of a given shape
in the cloud would effectively alter their scattering properties. For
observations at nadir, particle orientation can increase the extinction of the
Large Plate Aggregate of up to $20\ \unit{\%}$ \citep{barlakas21}. To first
order, the increase in extinction can be expected to cause a similar
overestimation of the retrieved IWC. This, however, is still considerably lower
than the differences observed due to different ice habits.

\section{Conclusions}
\label{sec:conclusions}

The main result from the experiments presented in this study is that we were
able to find two ice particle shapes, the Large Plate Aggregate and the 6-Bullet
Rosette, for which the results of the combined retrieval were consistent with
the observations as well as the in situ measured IWC and PSDs for flight B984.
Considering the co-location issues that likely affected the other two flights,
we interpret this as a cautious indication of the validity of the retrieval
implementation. Since the ARTS radiative transfer model and optical properties
from the ARTS single-scattering database constitute a crucial component of the
retrieval, this result also indicates that they work reliably across the
millimeter- and sub-millimeter domain.

The results confirm the simulation-based findings from \citep{pfreundschuh20},
that a synergistic retrieval based on active and passive microwave observations
can help to better characterize the PSD of large ice hydrometeors ($D_\text{MAX}
> 200\ \unit{\mu m}$) than a radar-only retrieval alone. This indicates that
such retrievals can be used to study the microphysical properties of clouds and
thus help to improve their representation in weather and climate models.

However, the retrieval is at the same time very sensitive to the assumed ice
particle habit that is used in the retrieval forward model. We found no evidence
of a signal that could help to constrain the ice particle shape based solely on
the combination of radar and microwave observations, not even when
sub-millimeter observations are included. This means that more work is needed to
better constrain the shape a priori or that even more observations must be
integrated into the retrieval.

Although further work will be required, this study demonstrates the feasibility
and potential of synergistic retrievals of ice hydrometeors by combining active
and passive observations at millimeter and sub-millimeter wavelengths. Since the
combined retrieval can better constrain the PSD of ice hydrometeors, it may be a
useful tool to study the representation of clouds in NWP and climate models.
Additionally, as illustrated in this study, the retrieval can be used to study
the representation of ice hydrometeors in radiative transfer simulations, which
will be vital to many applications of observations from upcoming sub-millimeter
sensors such as ICI and the Arctic Weather Satellite
\citep{arctic_weather_satellite}.


%% The following commands are for the statements about the availability of data sets and/or software code corresponding to the manuscript.
%% It is strongly recommended to make use of these sections in case data sets and/or software code have been part of your research the article is based on.

\codeavailability{All code used to produce the results in this study is available through public repositories \citep{mcrf, ismar_combined}.} %% use this section when having only software code available


\dataavailability{A detailed listing of the datasets that were used in this study
 together with their sources is provided in Tab.~\ref{tab:data}.
 }





} %% use this section when having only data sets available



\appendix




%% Regarding figures and tables in appendices, the following two options are possible depending on your general handling of figures and tables in the manuscript environment:

%% Option 1: If you sorted all figures and tables into the sections of the text, please also sort the appendix figures and appendix tables into the respective appendix sections.
%% They will be correctly named automatically.

%% Option 2: If you put all figures after the reference list, please insert appendix tables and figures after the normal tables and figures.
%% To rename them correctly to A1, A2, etc., please add the following commands in front of them:

\appendixfigures
\begin{figure}
  \centering
  \includegraphics[width=1.0\textwidth]{figures/fig16}
  \caption{
    Scatter plots of retrieved IWP and corresponding residual in the fitted
    observations $325 \pm 3.5 \unit{GHz}$ ISMAR channel. Each column displays
    the residual distributions for the five different particle habits.}
  \label{fig:residuals_scatter}
\end{figure}


\noappendix       %% use this to mark the end of the appendix section

\appendixtables   %% needs to be added in front of appendix tables

%% Please add \clearpage between each table and/or figure. Further guidelines on figures and tables can be found below.



\authorcontribution{Simon Pfreundschuh has performed the retrieval calculations
  and data analysis as well as written the manuscript. Patrick Eriksson, Stefan
  A. Buehler, Manfred Brath, David Duncan and Simon Pfreundschuh have
  collaborated on the study that lead to the development of the presented
  algorithm. Stuart Fox, Richard Cotton, Florian Ewald have provided the flight
  campaign data, guidance regarding their usage and contributed to the
  interpretation and discussion of the retrieval results.}

\competinginterests{No competing interests are present} %% this section is mandatory even if you declare that no competing interests are present

\begin{acknowledgements}

The work of SP and PE on this study was financially supported by the Swedish
National Space Agency (SNSA) under grants 150/14 and 166/18.

SB was supported by the Deutsche Forschungsgemeinschaft (DFG, German Research
Foundation) under Germany's Excellence Strategy --- EXC 2037 'Climate, Climatic
Change, and Society' --- Project Number: 390683824, contributing to the Center
for Earth System Research and Sustainability (CEN) of Universit\"{a}t Hamburg.
 
SB’s work contributes to the Cluster of Excellence “CLICCS—Climate,
Climatic Change, and Society” funded by the Deutsche Forschungsgemeinschaft DFG
(EXC 2037, Project Number 390683824), and to the Center for Earth System Research
and Sustainability (CEN) of Universität Hamburg.”

The computations for this study were performed using several freely available
programming languages and software packages, most prominently the Python
language \citep{python}, the IPython computing environment \citep{ipython}, the
numpy \citep{numpy}, pandas \citep{pandas} and xarray \citep{xarray} packages
for numerical computing, the satpy package \citep{satpy} for processing of
satellite data and matplotlib for generating figures \citep{matplotlib}.

The computations were performed on resources at Chalmers Centre for
Computational Science and Engineering (C3SE) provided by the Swedish National
Infrastructure for Computing (SNIC).

\citet{era5} was downloaded from the Copernicus Climate Change Service (C3S)
Climate Data Store.

The results contain modified Copernicus Climate Change Service information 2021.
Neither the European Commission nor ECMWF is responsible for any use that may be
made of the Copernicus information or data it contains.

HALO data are from SPP 1294 “High Altitude and Long Range Research Aircraft
(HALO)”, funded by the German Research Foundation (Deutsche
Forschungsgemeinschaft - DFG), project number 316646266.

\end{acknowledgements}




%% REFERENCES

%% The reference list is compiled as follows:


%% Since the Copernicus LaTeX package includes the BibTeX style file copernicus.bst,
%% authors experienced with BibTeX only have to include the following two lines:

\bibliographystyle{copernicus}
\bibliography{references}
%%
%% URLs and DOIs can be entered in your BibTeX file as:
%%
%% URL = {http://www.xyz.org/~jones/idx_g.htm}
%% DOI = {10.5194/xyz}


%% LITERATURE CITATIONS
%%
%% command                        & example result
%% \citet{jones90}|               & Jones et al. (1990)
%% \citep{jones90}|               & (Jones et al., 1990)
%% \citep{jones90,jones93}|       & (Jones et al., 1990, 1993)
%% \citep[p.~32]{jones90}|        & (Jones et al., 1990, p.~32)
%% \citep[e.g.,][]{jones90}|      & (e.g., Jones et al., 1990)
%% \citep[e.g.,][p.~32]{jones90}| & (e.g., Jones et al., 1990, p.~32)
%% \citeauthor{jones90}|          & Jones et al.
%% \citeyear{jones90}|            & 1990



%% FIGURES

%% When figures and tables are placed at the end of the MS (article in one-column style), please add \clearpage
%% between bibliography and first table and/or figure as well as between each table and/or figure.


%% ONE-COLUMN FIGURES

%%f
%\begin{figure}[t]
%\includegraphics[width=8.3cm]{FILE NAME}
%\caption{TEXT}
%\end{figure}
%
%%% TWO-COLUMN FIGURES
%
%%f
%\begin{figure*}[t]
%\includegraphics[width=12cm]{FILE NAME}
%\caption{TEXT}
%\end{figure*}
%
%
%%% TABLES
%%%
%%% The different columns must be seperated with a & command and should
%%% end with \\ to identify the column brake.
%
%%% ONE-COLUMN TABLE
%
%%t
%\begin{table}[t]
%\caption{TEXT}
%\begin{tabular}{column = lcr}
%\tophline
%
%\middlehline
%
%\bottomhline
%\end{tabular}
%\belowtable{} % Table Footnotes
%\end{table}
%
%%% TWO-COLUMN TABLE
%
%%t
%\begin{table*}[t]
%\caption{TEXT}
%\begin{tabular}{column = lcr}
%\tophline
%
%\middlehline
%
%\bottomhline
%\end{tabular}
%\belowtable{} % Table Footnotes
%\end{table*}
%
%%% LANDSCAPE TABLE
%
%%t
%\begin{sidewaystable*}[t]
%\caption{TEXT}
%\begin{tabular}{column = lcr}
%\tophline
%
%\middlehline
%
%\bottomhline
%\end{tabular}
%\belowtable{} % Table Footnotes
%\end{sidewaystable*}
%
%
%%% MATHEMATICAL EXPRESSIONS
%
%%% All papers typeset by Copernicus Publications follow the math typesetting regulations
%%% given by the IUPAC Green Book (IUPAC: Quantities, Units and Symbols in Physical Chemistry,
%%% 2nd Edn., Blackwell Science, available at: http://old.iupac.org/publications/books/gbook/green_book_2ed.pdf, 1993).
%%%
%%% Physical quantities/variables are typeset in italic font (t for time, T for Temperature)
%%% Indices which are not defined are typeset in italic font (x, y, z, a, b, c)
%%% Items/objects which are defined are typeset in roman font (Car A, Car B)
%%% Descriptions/specifications which are defined by itself are typeset in roman font (abs, rel, ref, tot, net, ice)
%%% Abbreviations from 2 letters are typeset in roman font (RH, LAI)
%%% Vectors are identified in bold italic font using \vec{x}
%%% Matrices are identified in bold roman font
%%% Multiplication signs are typeset using the LaTeX commands \times (for vector products, grids, and exponential notations) or \cdot
%%% The character * should not be applied as mutliplication sign
%
%
%%% EQUATIONS
%
%%% Single-row equation
%
%\begin{equation}
%
%\end{equation}
%
%%% Multiline equation
%
%\begin{align}
%& 3 + 5 = 8\\
%& 3 + 5 = 8\\
%& 3 + 5 = 8
%\end{align}
%
%
%%% MATRICES
%
%\begin{matrix}
%x & y & z\\
%x & y & z\\
%x & y & z\\
%\end{matrix}
%
%
%%% ALGORITHM
%
%\begin{algorithm}
%\caption{...}
%\label{a1}
%\begin{algorithmic}
%...
%\end{algorithmic}
%\end{algorithm}
%
%
%%% CHEMICAL FORMULAS AND REACTIONS
%
%%% For formulas embedded in the text, please use \chem{}
%
%%% The reaction environment creates labels including the letter R, i.e. (R1), (R2), etc.
%
%\begin{reaction}
%%% \rightarrow should be used for normal (one-way) chemical reactions
%%% \rightleftharpoons should be used for equilibria
%%% \leftrightarrow should be used for resonance structures
%\end{reaction}
%
%
%%% PHYSICAL UNITS
%%%
%%% Please use \unit{} and apply the exponential notation


\end{document}
