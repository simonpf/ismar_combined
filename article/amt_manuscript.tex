%% Copernicus Publications Manuscript Preparation Template for LaTeX Submissions
%% ---------------------------------
%% This template should be used for copernicus.cls
%% The class file and some style files are bundled in the Copernicus Latex Package, which can be downloaded from the different journal webpages.
%% For further assistance please contact Copernicus Publications at: production@copernicus.org
%% https://publications.copernicus.org/for_authors/manuscript_preparation.html


%% Please use the following documentclass and journal abbreviations for discussion papers and final revised papers.

%% 2-column papers and discussion papers
\documentclass[journal abbreviation, manuscript]{copernicus}



%% Journal abbreviations (please use the same for discussion papers and final revised papers)


% Advances in Geosciences (adgeo)
% Advances in Radio Science (ars)
% Advances in Science and Research (asr)
% Advances in Statistical Climatology, Meteorology and Oceanography (ascmo)
% Annales Geophysicae (angeo)
% Archives Animal Breeding (aab)
% ASTRA Proceedings (ap)
% Atmospheric Chemistry and Physics (acp)
% Atmospheric Measurement Techniques (amt)
% Biogeosciences (bg)
% Climate of the Past (cp)
% DEUQUA Special Publications (deuquasp)
% Drinking Water Engineering and Science (dwes)
% Earth Surface Dynamics (esurf)
% Earth System Dynamics (esd)
% Earth System Science Data (essd)
% E&G Quaternary Science Journal (egqsj)
% Fossil Record (fr)
% Geochronology (gchron)
% Geographica Helvetica (gh)
% Geoscience Communication (gc)
% Geoscientific Instrumentation, Methods and Data Systems (gi)
% Geoscientific Model Development (gmd)
% History of Geo- and Space Sciences (hgss)
% Hydrology and Earth System Sciences (hess)
% Journal of Micropalaeontology (jm)
% Journal of Sensors and Sensor Systems (jsss)
% Mechanical Sciences (ms)
% Natural Hazards and Earth System Sciences (nhess)
% Nonlinear Processes in Geophysics (npg)
% Ocean Science (os)
% Primate Biology (pb)
% Proceedings of the International Association of Hydrological Sciences (piahs)
% Scientific Drilling (sd)
% SOIL (soil)
% Solid Earth (se)
% The Cryosphere (tc)
% Web Ecology (we)
% Wind Energy Science (wes)


%% \usepackage commands included in the copernicus.cls:
%\usepackage[german, english]{babel}
%\usepackage{tabularx}
%\usepackage{cancel}
%\usepackage{multirow}
%\usepackage{supertabular}
%\usepackage{algorithmic}
%\usepackage{algorithm}
%\usepackage{amsthm}
%\usepackage{float}
%\usepackage{subfig}
%\usepackage{rotating}

\usepackage{todonotes}
\begin{document}

\title{Synergistic radar and sub-millimeter radiometer retrievals
  of ice hydrometeors in mid-latitude frontal cloud systems}

% \Author[affil]{given_name}{surname}

\Author[1]{Simon}{Pfreundschuh}
\Author[4]{Stuart}{Fox}
\Author[1]{Patrick}{Eriksson}
\Author[2]{Stefan A.}{Buehler}
\Author[2]{Manfred}{Brath}
\Author[3]{David}{Duncan}
\Author[5]{Florian}{Ewald}
\Author[6]{Julien}{Delanoë}

\affil[1]{Department of Space, Earth and Environment, Chalmers University of Technology, 41296 Gothenburg, Sweden}
\affil[2]{Meteorologisches Institut, Fachbereich Geowissenschaften, Centrum für Erdsystem und Nachhaltigkeitsforschung (CEN), Universität Hamburg, Bundesstraße 55, 20146 Hamburg, Germany}
\affil[3]{ECMWF, Shinfield Park, Reading RG2 9AX, United Kingdom}
\affil[4]{Met Office, FitzRoy Road, Exeter, EX1 3PB, United Kingdom}
\affil[5]{Institut für Physik der Atmosphäre, Deutsches Zentrum für Luft- und Raumfahrt, Münchener Straße 20, 82234 Oberpfaffenhofen-Wessling}
\affil[6]{Laboratoire Atmosphere, Milieux, et Observations Spatiales, Guyancourt, France}

\runningtitle{Synergistic radar and sub-millimeter radiometer retrievals
  of ice hydrometeors}
\runningauthor{Simon Pfreundschuh}
\correspondence{Simon Pfreundschuh (simon.pfreundschuh@chalmers.se)}

\received{}
\pubdiscuss{} %% only important for two-stage journals
\revised{}
\accepted{}
\published{}

%% These dates will be inserted by Copernicus Publications during the typesetting process.

\firstpage{1}

\maketitle

\begin{abstract}
  
  Accurate measurements of concentrations of ice hydrometeors are required to
  improve the representation of cloud and precipitation in weather and climate
  models. However, their variability in shape, size and concentration causes
  large uncertainties in currently available datasets. To improve measurements
  of ice hydrometeors, a synergistic ice cloud retrieval has been developed in a
  previous study that combines radar with millimeter and sub-millimeter
  radiometer observations in order to better constrain hydrometeor
  distributions.

  This study presents results of the retrieval algorithm applied to airborne
  observations of three mid-latitude, frontally-generated cloud systems. Good
  agreement with in-situ-measured ice water content and particle concentrations
  for particle maximum diameter over $200\ \unit{\mu m}$ is found for the flight
  for which observations could be confidently co-located with in-situ
  measurements. The variational retrieval manages to fit the observations well
  for all considered flights, which serves as a validation of the sub-millimeter
  radiative transfer modeling in cloudy atmospheres. Moreover, we find that a
  good fit to the observations can be obtained even for ice particle shapes that
  yield retrieval results that inconsistent with in-situ measurements. This
  indicates that even the combination of a wide range of cloud-sensitive
  microwave observations can not constrain the ice particle shapes in the
  observed clouds. Despite this, our results still show an improved sensitivity
  of the synergistic results to the cloud-microphysical properties of the cloud
  compared radar-only retrievals.

  This study validates the retrieval implementation and the underlying radiative
  transfer calculation. The results indicate that such a synergistic retrieval
  can indeed help to improve the accuracy of ice-hydrometeor retrievals in
  flight campaigns or as part a potential future satellite mission. The good
  fits obtained to the observations further increase confidence in the modeling
  of cloudy-sky radiative transfer in the ARTS radiative transfer model and the
  scattering properties of the ARTS single scattering database. This is an
  important indication that these tools can be used to produce accurate
  simulations of observations of upcoming space-borne sensors providing
  sub-millimeter observations of clouds such as the Ice Cloud Imager or the
  radiometer on the Arctic Weather Satellite.
 
 
\end{abstract}


\introduction  %% \introduction[modified heading if necessary]

The representation of clouds in climate models remains an important issue that
causes significant uncertainties in their predictions \citep{zelinka20}.
Improving and validating these models requires measurements that accurately
characterize the distribution of hydrometeors in the atmosphere. At regional and
global scale, such observations can be obtained efficiently only through remote
sensing. Unfortunately, currently available remote-sensing observations do not
constrain the distribution of ice in the atmosphere well \citep{waliser09,
  eliasson11, duncan18a}.

To address this, the Ice Cloud Imager (ICI) radiometer, which will be launched
onboard the second generation of European operational meteorological satellites,
will be the first operational sensor to provide global observations of clouds at
microwave frequencies exceeding 183 GHz. Compared to microwave observations at
currently available frequencies ($\leq 183\ \unit{GHz}$), observations at and
above $243\ \unit{GHz}$ have been shown to exhibit greater sensitivity to small
hydrometeors \citep{buehler12} as well as their shape and particle size
distribution \citep{evans98}. This increased sensitivity to small particles is
expected to help improve measurements of the distribution of ice in the
atmosphere but comes at the cost of increased complexity of the radiative
transfer due to the sensitivity to the microphysical properties of the particles
in the cloud.

In \citet{pfreundschuh19}, we have developed a cloud-ice retrieval that combines
radar measurements and passive sub-millimeter observations with the aim of
investigating the potential benefits of a synergistic radar mission to fly in
constellation with ICI on MetOp-SG. The simulation-based results from
\citet{pfreundschuh19} indicate that combining active and passive observations
across millimeter and sub-millimeter observations can indeed help to better
constrain the distributions of ice hydrometers in cloud retrievals. The
principal aim of this study is to validate the synergistic retrieval using real
observations and to investigate its potential applications in flight campaigns
to measure cloud properties. The retrieval is applied to radar and microwave
radiometer observations of three mid-latitude cloud systems and, for two of the
flights, compared to in-situ measurements of cloud bulk properties and particle
size distributions. Since the shape of ice particles is difficult to constrain a
priori, the retrieval is run multiple times while the ice particle shape assumed
in the radiative transfer simulations is varied.

Since the variational retrieval also fits a radiative transfer model to the
observations, the observations from the three flights also constitute an
opportunity to validate the radiative transfer through clouds at sub-millimeter
wavelengths. Accurate simulations of observations of clouds are of paramount
importance not only for future cloud retrievals from ICI observations
\citep{eriksson20} but also for assimilating cloud-contaminated observations in
numerical weather prediction models \citep{geer17}.

So far, direct efforts to validate sub-millimeter radiative transfer through
clouds at frequencies exceeding $183\ \unit{GHz}$ have been limited by the
availability of such observations and the difficulty of accurately representing
the observed cloud and atmosphere in the simulations. In the upper troposphere,
observations of clouds have been obtained as by-products of space-borne
limb-sounding missions aimed to study gases in the stratosphere. Retrievals of
ice mass concentrations using sub-millimeter limb-sounding observations were
developed for AURA MLS \citep{wu06}, Odin/SMR \citep{eriksson07} and SMILES
\citep{millan13, eriksson14}. However, due to their observation geometry the
observations are limited very high clouds in the tropics.

Other sources of sub-millimeter observations of clouds are airborne sensors with
sub-millimeter channels such as the Millimeter-wave imaging radiometer (MIR,
\citet{wang01}), the Compact Scanning Submillimeter Imaging Radiometer (CoSSIR,
\citet{evans05}) and the International Sub-Millimeter Airborne Radiometer
(ISMAR, \citet{fox17}). To the best knowledge of the authors, there are so far
only two notable efforts to validate microphysical assumptions in radiative
transfer simulations: Firstly, the study by \citet{evans05}, who use a Bayesian
retrieval to retrieve integrated radar backscatter of a convective anvil and
achieve good agreement with simultaneous radar measurements at $90\ \unit{GHz}$.
And secondly, the study by \citet{fox18}, who perform a closure study based on
lidar and in-situ observations of hydrometeors in a mid-latitude cirrus cloud.

The observations used in this study consist of flights during which the ISMAR
radiometer, which is the only sub-millimeter radiometer that is currently
operational, performed co-located measurements with a cloud radar. The
simultaneous availability of radar and passive sub-millimeter observations
allows the applications of a synergistic retrieval which aims to retrieve an
atmospheric state that is as close as possible to the observations. Compared to
other validation approaches this has the advantage of reducing uncertainties
arising from having to infer hydrometeor distributions and background of the
atmosphere from other sources. Assessing the how well the retrieval fits the
observations and the consistency of the retrieved and in-situ-measured
hydrometeor distributions can provide insight regarding the consistency of the
Atmospheric Radiative Transfer Simulator (ARTS) and the ARTS Single Scattering
Database (SSDB), which are used to implement the forward model used in the
retrieval.

To summarize, the two aims of this study are
\begin{enumerate}
 \item to validate the retrieval algorithm presented in \citep{pfreundschuh19} and
 \item to validate the representation of cloud microphysical properties in the
   ARTS radiative transfer model and the ARTS single scattering database
\end{enumerate}
The remainder of this article is structured as follows:
Section~\ref{sec:methods_and_data} contains a description of the data as well as
the retrieval algorithm that is applied to the remote sensing observations.
Section~\ref{sec:results} presents the results of the retrieval as well as the
comparisons to in-situ data followed by a discussion of those result in
Section~\ref{sec:discussion} and conclusions in~\ref{sec:conclusions}.


\section{Data and methods}
\label{sec:methods_and_data}

\begin{figure}[h!]
  \centering \includegraphics[width=0.9\textwidth]{../plots/flight_overview}
  \caption{Flight paths of the cloud overpasses considered in this study. First
    row of panels shows the true-color composite derived from the closest
    overpass of the MODIS sensor onboard the Aqua satellite. Second row shows
    ERA5 temperature (colored background), geopotential (contours) and wind
    speed (arrows) at the $800\ \unit{mb}$ pressure level.}
  \label{fig:flight_paths}
\end{figure}

The combined retrieval algorithm used in this study retrieves distributions of
ice hydrometeors by combining radar and microwave radiometer observations. The
passive observations are taken from the the MARSS \citep{mcgrath01} and ISMAR
\citep{fox17} radiometers on board the FAAM BAe-146 aircraft. The
instrumentation of the FAAM aircraft does not include a cloud radar which is why
only flights for which the radiometer observations can be co-located with radar
observations from another platform are suitable to apply the combined retrieval
to. Since ISMAR is currently the only operational radiometer with channels up to
$664$ and $874\ \unit{GHz}$, the flights considered in this study provide a rare
opportunities to study the synergies between radar and passive (sub-)millimeter
observations using real observations.

An overview of the three flights and the corresponding meteorological contexts
is provided in Fig.~\ref{fig:flight_paths}. The first considered flight,
designated B984, took place on 14 October 2016 as part of the North Atlantic
Waveguide and Downstream Impact Experiment (NAWDEX, \citet{schafler18}). During
this flight, a cloud system generated by an occluded front has been observed
quasi-simultaneously by three research aircraft: The High Altitude and Long
Range Research Aircraft (HALO, \citet{krautstrunk12}), the Facility for Airborne
Atmospheric Measurements (FAAM) and the Service des Avions Francais
Instrumentations pour la Recherche en Environnement (SAFIRE).

The two other flights, designated C159 and C161 respectively, took place in
March 2019 as part of the PIK'N'MIX campaign. These two flights were performed
following the ground track of simultaneous overpasses of the CloudSat satellite.
The observations probe clouds in different regions of a frontal system generated
by a low-pressure system passing over Iceland around the 21st of March 2019. The
cloud system observed during flight C159 is a stratiform, lightly-precipitating
cloud located in the warm sector of the fronal system, whereas the clouds
observed during flight C161 are of convective origin and located in the active
region of the cold front.


\subsection{Radar observations}

The radar observations from the three flights are displayed in
Fig~\ref{fig:observations_radar}. The radar observations for flight B984 stem
from the HAMP MIRA radar \citep{ewald19} onboard the HALO aircraft. The radar
operates at a frequency of $35\ \unit{GHz}$. Its observations have been
downsampled to a vertical resolution of $210\ \unit{m}$ and a horizontal
resolution of roughly $700\ \unit{m}$ in order to reduce the computational
complexity of the retrieval.

The radar observations for flights C159 and C161 stem from the CloudSat Cloud
Profiling Radar (CPR, \citet{tanelli08}), which operates at $94\ \unit{GHz}$.
Since the CloudSat observations were affected  strongly by ground-clutter,
the first five bins located completely above surface altitude were set to the
reflectivity found in the sixth bin above the surface.

\begin{figure}[h!]
  \centering
  \includegraphics[width=0.8\textwidth]{../plots/observations_radar.png}
  \caption{Radar observations from the flights used in this study. Panel (a)
    shows the radar reflectivity measured by the HAMP MIRA $35\ \unit{GHz}$
    cloud radar. Panels (b) and (c) show the reflectivity measured by the
    CloudSat CPR at $94\ \unit{GHz}$.}
  \label{fig:observations_radar}
\end{figure}

\subsubsection{MARSS}

The MARSS radiometer measures microwave radiances at $89, 157$ and channels
located around the water vapor line at $183 \unit{GHz}$. Although MARSS is a
scanning radiometer only observations within $5\ \unit{^\circ}$ of nadir are
used in the retrieval. The observations from the three flights are displayed in
Fig.~\ref{fig:observations_marss}. Flight sections above land surfaces have been
highlighted in the plots because observations that are strongly affected by
surface emissions are not used in the retrieval. For the MARSS sensor this is
the case for the window channels at $89$ and $157\ \unit{GHz}$.

\begin{figure}[h!]
  \centering
  \includegraphics[width=1.0\textwidth]{../plots/observations_marss.png}
  \caption{
    Passive microwave measurements from the MARSS radiometer together with the
    matched radar observations. Grey background in the radiance plots marks
    observations that were taken over land and are therefore not used in
    the retrieval.
    }
  \label{fig:observations_marss}
\end{figure}


\subsubsection{ISMAR}

\begin{figure}[h!]
  \centering
  \includegraphics[width=1.0\textwidth]{../plots/observations_ismar.png}
  \caption{
    Passive microwave measurements from the ISMAR radiometer together with the
    matched radar observations. Grey background in the radiance plots marks
    observations that were taken over land and are therefore not used in
    the retrieval.
    }
  \label{fig:observations_ismar}
\end{figure}

The ISMAR radiometer has channels covering the frequency range from
$118\ \unit{GHz}$ up to $874 \unit{GHz}$. Also from ISMAR only observations
within five degrees off nadir are used here. The observations from the 3 flights
are displayed in Fig.~\ref{fig:observations_ismar}. Similar as for the two
lowest MARSS channels, the 4 outermost channels around the $118\ \unit{GHz}$
oxygen line are not used in the retrieval for observations over land.
Unfortunately, not all channels were available on all flights. The channels
around $448$ and $874\ \unit{GHz}$ were not available on the B984 flight, while
the two of the channels around $325\ \unit{GHz}$ were missing for the C151 and
C161 flights.


\subsection{In-situ measurements}
\label{sec:in_situ}

In-situ measurements of cloud hydrometeors were performed during  the flights
B984 and C159. A detailed view of the high-level runs and the corresponding
in-situ sampling paths are provided in Fig.~\ref{fig:flight_paths_detail}.
For flight C159, this view  reveals a noticeable offset not only between the
high-level run and the in-situ sampling path, but also between ground tracks
of radar and radiometer observations.


\begin{figure}[h!]
  \centering
  \includegraphics[width=1.0\textwidth]{../plots/flight_overview_detail.png}
  \caption{
    Detailed view of the flight paths of the high-level runs and in-situ sampling
    paths for flights B984 and C159. Background is the same as in
    Fig.~\ref{fig:flight_overview}.
    }
  \label{fig:flight_oberview_detail}
\end{figure}

The in-situ measurements that are relevant to this study are bulk ice water
content measured using a Nevzorov hot-wire probe \citep{korolev13} and PSDs
recorded using DMT CIP-15 and DMT CIP-100 probes, which measure size-resolved
particle concentrations with resolutions of 15 and $100\ \unit{\mu m}$,
respectively. The in-situ measurements were mapped to corresponding radar
observations using a nearest-neighbor criterion. An overview over the measured
IWC and PSDs is provided in Fig.~\ref{fig:in_situ}.

\begin{figure}[hbpt!]
  \centering
  \includegraphics[width=0.9\textwidth]{../plots/in_situ_measurements.png}
  \caption{
    In-situ-measured IWC and PSDs for flights B984 and C159. The row displays
    the measured IWC along the flight path plotted on top of the co-located radar
    observations. Subsequent rows display the mean of the measured PSDs (red)
    together with randomly sampled instances (light-grey) over different
    altitudes.
    }
  \label{fig:in_situ}
\end{figure}

\subsection{Retrieval algorithm}
\label{sec:synergistic_retrieval}

The synergistic retrieval algorithm used in this study is based on the optimal
estimation framework \citep{rodgers00} and retrieves distributions of frozen and
liquid hydrometeors by simultaneously fitting a forward model to the active and
passive observations. The OEM-based cloud-retrieval algorithm is described in
detail in \citet{pfreundschuh19} and a length description is therefore omitted
here.

The algorithm has been adjusted to the flight measurements by adapting the
forward model to the sensors that were available for each flight. Furthermore,
the atmospheric grid was limited to altitudes between $0$ and $10 \unit{km}$ and
matched to the resolution of the radar observations. Background properties of
atmosphere and surface, such as temperature and wind speed, as well as a priori
profiles for relative humidity and liquid cloud water are taken from the ERA5
hourly reanalysis.

The latest stable release (version 2.4) of the Atmospheric Radiative Transfer
Simulator (ARTS, \citet{arts18}) is used to implement the forward model used in
the retrieval. Radar reflectivities and passive radiances are calculated using
ARTS's built-in radar solver and the ARTS interface to DISORT \citep{disort00},
both of which provide approximate analytical Jacobians. Polarization is
neglected in all simulations.

\subsubsection{Representation of frozen hydrometeors}

The forward model simulates the active and passive observations in two steps: In
the first one, the hydrometeor bulk properties that are used to describe
hydrometeor distribution in the atmosphere are mapped to corresponding optical
properties. These are then, in the second step, used together with background
atmosphere and surface to simulate the observations. As the ratio between
hydrometeor size and observation wavelength increases, the optical properties
become more sensitive to the microphysical properties of the hydrometeors such
as their shape and size distribution. Since both particle shape and distribution
of ice particles vary extensively, this causes significant uncertainties in
mapping from hydrometeor bulk properties to optical properties especially for
ice particles observed at sub-millimeter wavelengths.

To map the hydrometeor bulk properties to optical properties, the forward model
follows the common approach based on a particle size distribution (PSD)
parametrized by the bulk properties and an ice particle model that maps particle
sizes to shapes and corresponding optical properties. More specifically, the
forward model uses the normalized PSD approach proposed by \citet{delanoe05},
which is based on a modified gamma distribution parametrized by mass-weighted
mean diameter ($D_m$) and intercept parameter ($N_0^*$). The updated values from
\citet{cazenave19} are used as shape parameters of the distribution. The PSD is
combined with a so called (particle) habit, which consists of a sequence of ice
particle shapes of varying sizes with pre-computed optical properties, that
allows mapping particle sizes to optical properties. Bulk optical properties are
then calculated by integrating the product of particle density and optical
properties over the particle size. As the retrieval is currently set up, the
particle habit cannot be retrieved and must be assumed a priori. Since the
choice of particle habit affects the retrieval results, a set of habits has been
chosen with which the retrieval will be run in order to investigate their
impact.

Five habits were chosen from the set of standard habits that are distributed
with the ARTS SSDB. The standard habits are particle mixes that combine pristine
crystals at small sizes with aggregate shapes at larger sizes. The selected
habits are listed in Tab.~\ref{tab:particle_habits}. An overview of their bulk
optical properties at the two frequencies of the cloud radars and at
$325\ \unit{GHz}$ is provided in Fig.~\ref{fig:particle_properties}. The
particles were selected so that their properties (colored lines) cover most of
the variability in the standard habits (grey lines).

\begin{figure}[hbpt!]
  \centering
  \includegraphics[width=0.9\textwidth]{../plots/particle_properties}
  \caption{Properties of the habits used to represent frozen hydrometeors
    in the retrieval. Colored lines display the properties of the habits that
    were used in the retrieval, while grey lines show the properties of the
    other standard habits distributed with the ARTS SSDB.
    }
  \label{fig:particle_properties}
\end{figure}

\begin{table}
  \centering
  \caption{Particle habits used in the retrieval. The mass size relationship is given
    in terms of the parameters
    of a fitted power law of the form $m = \alpha \cdot D_\text{max}^\beta$ with
    $D_\text{max}$ the maximum diameter and $m$ in $\unit{kg\ m^{-3}}$.}
  \begin{tabular}{l|l|rr|rr}
    \multicolumn{1}{c|}{Name} & \multicolumn{1}{c|}{Shapes used} &
    \multicolumn{2}{c|}{Size range} & \multicolumn{2}{c}{Mass size relationship}
    \\
    & Name (ID) &$D_{\text{eq}, \text{ min}}\ [\unit{\mu m}$] &
    $D_{\text{eq}, \text{ max}}\ [\unit{\mu m}]$ &\hfill
    $\alpha$ & \hfill $\beta$ \\
    \hline \hline % GEM
    %
    % Large Plate Aggregate
    %
    8-Column Aggregate & 8-Column Aggregate (11) & $10\ $ & $3000\ $ & \hfill 440 & \hfill 3 \\

    %
    % Evans snow aggregate
    %
    Evans Snow Aggregate & Evans Snow Aggregate (11) & $10\ $ & $3000\ $ & \hfill 440 & \hfill 3 \\

    %
    % Large Plate Aggregate
    %
    \hline Large Plate Aggregate & Thick Plate (15) & $16\ $ & $200\ $ & \hfill
    110 & \hfill 3 \\ & Large Plate Aggregate (33) & $160\ $ & $3021\ $ & \hfill
    0.21 & \hfill 2.26 \\
    %
    % Large Column Aggregate
    %
    \hline Large Column Aggregate & Block Column (12) & $10\ $ & $200\ $ &
    \hfill 110 & \hfill 3 \\ & Large Column Aggregate (22) & $160\ $ & $3021\ $
    & \hfill 0.25 & \hfill 2.43 \\
    %
    % 8-Column Aggregate
    %
    \hline Large Column Aggregate & Block Column (12) & $10\ $ & $200\ $ &
    \hfill 110 & \hfill 3 \\ & Large Column Aggregate (22) & $160\ $ & $3021\ $
    & \hfill 0.25 & \hfill 2.43 \\
    & Name (ID) &$D_{\text{eq}, \text{ min}}\ [\unit{\mu m}$] &
    $D_{\text{eq}, \text{ max}}\ [\unit{\mu m}]$ &\hfill
    $\alpha$ & \hfill $\beta$ \\
    \hline \hline % GEM
    CloudIce & GEM CloudIce (11) & $10\ $ & $3000\ $ & \hfill 440 & \hfill 3 \\
  \end{tabular}
  \label{tab:particle_properties}
\end{table}

Complementary information that can help guide the selection of a suitable
particle shape can be obtained from in-situ measurements. Since the particle
habit associates particle sizes with a specific shape it can be used to compute
a bulk water content corresponding to PSD measurements. This allows calculating
the IWC corresponding to the in-situ-measured PSDs, which can be compared with
the IWC measured using the Nevzorov probe. The agreement between the PSD-derived
IWC and the directly measured IWC can then provide insight into whether the
mass-size relation corresponding to the particle shape is consistent with the
in-situ measurements. Such a comparison is provided in
Fig.~\ref{fig:mass_size_relation}.

For both flights the large plate aggregate shape and the sector snowflake agree
best with the directly measured IWC. The large column aggregate yields values at
the low end of the measured distribution for all flights and altitudes. The
Evans snow aggregate overestimates the IWC for high altitudes but underestimates
it a low altitudes for flight B984, while for C159 the mass is consistently
underestimated. The 8-column aggregate underestimates IWC for high altitudes but
overestimates it for medium and low altitudes for flight B984 while it
consistently overestimates the mass for flight C159.

\begin{figure}
  \centering
  \includegraphics[width=1.0\textwidth]{../plots/mass_size_relations.png}
  \caption{
    Comparison of bulk IWC content as measured by the Nevzorov probe and
    inferred from in-situ-measured PSDs using a given particle shape. The
    background in each plot shows the distribution of Nevzorov-measured bulk
    ice water content for a given $1\ \unit{km}$-altitude bin. Colored boxes
    display the distribution of the mass distributions in that bin inferred
    using a given particle shape.
    }
  \label{fig:mass_size_relation}
\end{figure}


\section{Results}
\label{sec:results}

While the primary results of the combined retrieval are certainly the retrieved
hydrometeor distributions, the retrieval also produces fitted observations whose
agreement with the true observations can provide valuable information regarding
the accuracy of the forward model and the fitness of a priori assumptions. These
results obtained by running the retrieval for each combination of flights and
selected particle habits are presented in this section. Additionally, the
retrieval results are compared to in-situ measurements for the two flight for
which those have been available.


\subsection{Fit to observations}

To provide a detailed view on the residuals, i.e. the difference between
simulated and real observations, for a specific habit, residuals obtained with
the large plate aggregate shape are displayed in Fig.~\ref{fig:residuals}. As
the figure shows, the retrieval was able to fit both radar and radiometer
observations fairly well for all flights. For flight B984, the radar residuals
show some scattered deviations located at the edge of the cloud, which are
likely discretization artifacts. Except for that, residuals for this flight
remain well within $1\ \unit{dB}$. For the two other flights more systematic
deviations are visible in the radar results but also those are mostly within a
few $\unit{dB}$.

Radiometer residuals for flight B984 are mostly within $\pm 5\ \unit{K}$. For
the two other flights residuals are larger. Especially the $247\ \unit{GHz}$
channel exhibits some regions with biases up to and exceeding $10\ \unit{K}$ for
flights C159 and C161. A potential explanation for this is the sensitivity of
this channel to liquid precipitation in the lower troposphere, which may lead to
significant biases when radar and radiometer observations are not well
co-located. In general, however, the biases do not exhibit any clear signs of
systematic misfit associated to the presence of hydrometeors.

\begin{figure}[!hbpt]
  \centering
  \includegraphics[width=0.8\textwidth]{../plots/retrieval_residuals}
  \caption{Differences between observed and fitted observations obtained
    using the large plate aggregate particle shape. The first row depicts
    the radar measurements for each flight. The second row show the residuals
    in the fitted radar observations. Following rows show the retrieval residual
    in the radiometer measurements for each of the frequency bands used in
    the retrieval.
    }
  \label{fig:residuals}
\end{figure}

For a more systematic analysis of the effect of the assumed particle shape on
the retrieval residuals, their distribution for radar and radiometer channels
around $183\ \unit{GHz}$ and above are displayed in
Fig.~\ref{fig:residuals_box}. The distributions, which for most channels the are
close to or centered around 0, confirm that the retrieval generally fits the
observations. The largest deviations are observed for the $874\ \unit{GHz}$
channel and the $243\ \unit{GHz}$ channel for flight C161. What is particularly
interesting is that the ice particle habit only has a minor impact on the
residuals indicating that the retrieval can compensate for mismatches in the
assumed particle shape by adjusting the retrieval variables.


\begin{figure}[!hbpt]
  \centering
  \includegraphics[width=0.8\textwidth]{../plots/residual_distributions}
  \caption{Distributions of retrieval residuals for different particle
    shapes used in the forward model for each of the three flights.}
  \label{fig:residuals_box}
\end{figure}

\subsection{Retrieved ice water content}

The retrieved bulk IWC and corresponding IWP for all three cloud scenes are
displayed in Fig.~\ref{fig:ice_water_content}. For all three flights, the ice
particle shape has a significant effect on the retrieved water content. In terms
of IWP, the large-column and evans snow aggregate yield the highest values
across all flights, while the 8-column aggregate consistently yields the lowest
IWP. The large plate aggregate and 6-bullet rosette both yield values within the
range of the other particle models, with the 6-bullet rosette leading to
slightly higher IWP values than the large plate aggregate. In addition to the
effect of the increased total retrieved water content, the particle habit also
has a small effect on the vertical distribution of the ice hydrometeors.


\begin{figure}[!hbpt]
  \centering
  \includegraphics[width=1.0\textwidth]{../plots/ice_water_content}
  \caption{Retrieved IWP and IWC content for all flights and
    different ice particle shapes assumed in the retrieval.}
  \label{fig:ice_water_content}
\end{figure}

\subsection{Comparison to in-situ measurements}

The most important question regarding the hydrometeor retrieval is certainly
whether the retrieved bulk properties are consistent with the in-situ
measurements. As mentioned in Sec.~\ref{sec:in_situ}, the in-situ measurements
were mapped to the radar observations using a nearest-neighbor criterion. For
B984, retrieval results within a distance of 1km of the flight path were then
associated to the in-situ measurements. Because of the mismatch between
observations and in-situ sampling paths, another approach was taken for flight
C159. Here the retrieval results were mapped to the in-situ measurements by
selecting all retrieval results between $50$ and $150\ \unit{km}$. Both,
the matched retrieval results and the vertically-resolved distributions of
measured and retrieved IWC are displayed in Fig.~\ref{fig:in_situ_iwc}.

For flight B984, the distribution of in-situ-measured IWC values is well
within the range of retrieved IWC values across all particle shapes up
to an altitude of around $7\ \unit{km}$. Below that altitude, the best
match to the in-situ measurements is achieved with the 6-bullet rosette
particle and the large plate aggregate. The 8-column aggregate underestimates
the in-situ-measured IWC while large column aggregate and evans snow aggregate
overestimate it. Above around $6\ \unit{km}$ all particles strongly
underestimate the in-situ-measured IWC. A likely cause for this is the
presence of a large number of small particles for which the microwave
observations lack sensitivity.

For flight C159 the distribution of retrieved IWC still covers the distribution
of in-situ-measured values but show a tendency towards underestimation. Overall,
difference between habits are smaller for this flight. However, the large
uncertainties caused by the large sampling region make these results less
conclusive.

\begin{figure}[!hbpt]
  \centering
  \includegraphics[width=1.0\textwidth]{../plots/in_situ_iwc}
  \caption{In-situ-measured and retrieved IWC for flights B984
    and C159. The first row of panels shows the in-situ sampling paths in
    relation to the measured radar reflectivity and marked by the
    blue shading the retrieval values that are mapped to the
    in-situ measurements. The second row of panels shows in the background
    the distribution of in-situ measured IWC values altitude bins with
    a height of $500\ \unit{m}$. Colored boxes display the corresponding
    distribution of retrieved IWC values for different ice particle shapes.}
  \label{fig:in_situ_iwc}
\end{figure}

Finally, we want to address the question whether the representation of cloud
microphysics within the retrieval forward model is consistent with the in-situ
measured distributions of hydrometeors. For this, we calculate the PSDs
corresponding to the retrieved bulk properties and compare the to the in-situ
measured PSDs. The results of the comparison are displayed in
Fig.~\ref{fig:in_situ_psds}. Similar as for the bulk water content the retrieved
PSDs match the in-situ measured ones quite well for flight B984 but show
systematic deviations for flight C159.

For flight B984, we find a good agreement between retrieved and in-situ-measured
PSDs for particles sizes $> 200 \ \unit{\mu m}$. The fact the this is observed
even at altitudes above $6 \ \unit{km}$ affirms that the underestimation of
IWC is caused by the presence of a large number of smaller particles.

\begin{figure}[!hbpt]
  \centering
  \includegraphics[width=1.0\textwidth]{../plots/in_situ_psds}
  \caption{In-situ-measured and retrieved PSDs for flights B984
    and C159. Each row of panels shows the mean of the in-situ-measured
    PSDs (black) together with randomly drawn samples of measured PSDs
    (light grey) for a given altitude bin of a height of one kilometer.
    Colored lines on top show the corresponding mean retrieved PSD for
    different assumed particle shapes.}
  \label{fig:in_situ_psds}
\end{figure}

\section{Discussion}
\label{sec:discussion}

This study used a variational retrieval to to retrieve vertically-resolved
distributions of ice hydrometeors from co-located radar and microwave radiometer
observations. For most of the considered channels, the retrieval succeeded in
fitting both the active and passive observations without significant, systematic
deviations. For two of the flights, the retrieved hydrometeor distributions
could be compared to in-situ measurements. For one of the flights (B984), we
found good agreement with the in-situ-measured bulk IWC for altitudes between
$2$ and $6\ \unit{km}$ for two of the particle shapes. The same particle shapes
also yielded the best agreements with the in-situ-measured PSDs for particle
sizes $D_\text{max} > 200\ \unit{\mu m}$. For the second flight (C159), no
consistency was found between the in-situ-measurements and the retrieval. A
likely explanation for this, however, is that the in-situ measurements weren't
as well co-located as for the first flight and that the cloud field was
potentially less homogeneous.

\subsection{Sub-millimeter radiative transfer in cloudy atmospheres}

 A first important result of this study is the ability of the retrieval to find
 atmospheric states that are consistent with the observed radiances and radar
 reflectivities for all three flights. This in itself is not self-evident due to
 the uncertainties that still affect the modeling of ice-particle scattering
 across millimeter and sub-millimeter wavelengths. Previous studies that tried
 to directly validate sub-millimeter RT through clouds were either limited to
 tropical clouds \citep{evans05, eriksson07} or cirrus clouds \citep{fox17}.

 For flight B984, the radar and all passive observations were fitted up to small
 systematic biases no larger than $3\ \unit{K}$. The deviations for the two
 other flights were generally larger, but these were likely caused by spatial
 and temporal co-location errors. The small residuals that were achieved for
 flight B984 indicate that both, the assumed optical properties as well as the
 retrieval forward model, are consistent across the simulated wavelengths.
 Moreover, the good agreement between real and simulated observations and
 between the retrieved and in-situ-measured hydrometeor distributions shows that
 the ARTS radiative transfer model can be used to perform accurate simulations
 of complex cloud scenes at millimeter and sub-millimeter wavelengths. This is
 an important result for future applications of sub-millimeter observations such
 as retrievals and data assimilation, which rely on such simulations.


\subsection{The impact of assumed ice particle shape}

A rather unexpected result that emerged from this study is that the retrieval
can fit the observations regardless of the assumed ice particle shape. This
indicates that although the observations are sensitive to variations in ice
particle shape, they alone can not constrain it. This is in agreement with what
has been reported in \citet{pfreundschuh19}, namely that no correlation could be
found between the particle shape yielding best retrieval fit and the one
yielding the most accurate retrieval results.

In an effort to better separate a potential signal from the ice particle shape
in the retrieval residuals, Fig.~\ref{fig:residuals_scatter} displays the
relation between IWP and residuals in the $325 \pm 3.5 \unit{GHz}$ channel. This
channel was chosen because it belongs to the channels displaying the largest
differences between residual distributions for difference particle habits in
Fig.~\ref{fig:residuals_box}. Nonetheless, the plots neither show no sign of a
relationship between neither residual and ice water content nor the residuals
across different habits.

This result has implications for cloud retrievals based on microwave
observations. These should either account for the uncertainty caused by
variations in ice particle shape or find ways to more accurately constrain the
shape a priori. Moreover, for studies that seek to validate model predictions by
comparing simulated and observed microwave observations, this implies that care
must be taken to accurately characterize the ice particle shape since
consistency between simulations and observations can be achieved for bulk water
contents that vary by almost an order of magnitude (c.f.
\ref{fig:ice_water_content}).

\subsection{Representation of cloud microphysics}

The lack of a signal that constrains the ice particle shape even in the combined
observations puts additional weights on the question of how to best represent
ice particles in simulations of microwave observations. The habits that yielded
the most accurate retrieval results in this study were the large plate aggregate
and the sector snow flake. However, these findings are based on observations
from the single flight for which the retrieval results could be reliably
compared with in-situ measurements. This result may be seen as indication that
these habits may work well for similar mid-latitude cloud systems. However, more
generally applicable conclusions would require further and more systematic
investigation.

\subsection{Retrieval validation}

Since the results presented in \citet{pfreundschuh19} were limited to
simulations based on a high-resolution climate model, the validation of the
retrieval using real observations remained an open issue. For flight, B984 good
agreement was found between retrieval results and in-situ measurements. Although
the retrieved IWC deviates from the in-situ measurements at altitudes $>
6\ \unit{km}$, the retrieved PSDs still match the in-situ measurements well for
particles with $D_\text{max} > 200\ \unit{\mu m}$. This indicates that the
observed deviations are due to the presence of a large number of small ice
particles that the microwave observations are not sensitive to. The presence of
a cloud layer with a large number of small particles was also reported by
\citet{ewald21} who investigated the same cloud system with combined radar-lidar
observations.

For flight C159 no agreement was found between retrieved and in-situ-measured
IWC and PSDs. However, multiple lines of evidence suggest that this is due to
co-location issues rather than failure of the retrieval algorithm. Firstly, the
flight path for the in-situ sampling was found to be offset from the high-level
run during which the observations were taken in the direction opposite to the
wind at $800\ \unit{mb}$ (Fig.~\ref{fig:flight_overview},
Fig.~\ref{fig:flight_overview_detail}). Secondly, there are visible disparities
between the in-situ measured IWC and the radar reflectivities even for IWC
values that should clearly appear in the RADAR measurements
(Fig.~\ref{fig:in_situ_iwc}).

\subsection{The added value synergistic cloud retrievals}

Although the evidence from flight B984 suggests that the synergistic
retrieval algorithm works well for retrieving ice hydrometeor concentrations,
similar retrievals can be performed using only radar observations. Especially
for airborne observations, such radar-only retrievals have the advantage of
being computationally much less complex since the effects of multiple
scattering can be neglected.

To investigate the potential added value of the combined retrieval for the
retrieval of ice hydrometeors, the results of the combined and an equivalent
radar-only retrieval for flight B984 are displayed in
Fig.~\ref{fig:in_situ_iwc_radar_only}. While the radar-only retrieval is affected to
lower degree by the choice of the combined and radar-only are largely the same
for altitudes between $3$ and $6\ \unit{km}$. Below that, the radar-only
retrieval tends to slightly overestimate the in-situ measurements, whereas the
combined retrieval matches the results in-situ measurements well but for two of
the ice particle shapes and even underestimate them for the 8-column aggregate.
At altitude above $6\ \unit{km}$ both retrievals underestimate the in-situ
measurements, with the combined retrieval showing larger spread between
particle habits.

\begin{figure}[!hbpt]
  \centering
  \includegraphics[width=1.0\textwidth]{../plots/in_situ_iwc_radar_only}
  \caption{In-situ-measured and retrieved PSDs for flight B984
    for the combined and the radar-only retrieval. Each panel displays
    the distributions of in-situ-measured IWC for different altitude
    bins in the background and the distribution of retrieved IWC values
    for different ice habits as colored boxes on top.}
  \label{fig:in_situ_iwc_radar_only}
\end{figure}

A similar comparison is shown in Fig.~\ref{fig:in_situ_psds_radar_only} for the
retrieved PSDs. The PSDs are similar for both retrievals for altitudes above
$4\ \unit{km}$. Below that, however, the radar-only retrieval over estimates the
particle concentrations, while the combined retrieval matches the in-situ
measurements quite well. This indicates that the combined retrieval utilizes the
complementary information in the radar and passive observations to match both
moments of the PSD, whereas the radar-only retrieval can
only match one of them.


\begin{figure}[!hbpt]
  \centering
  \includegraphics[width=1.0\textwidth]{../plots/in_situ_psds_radar_only}
  \caption{In-situ-measured and retrieved PSDs for flight B984
    retrieved using the combined (panel (a)) and the radar-only retrieval
    (panel (b)). Each row of panels shows the mean of the in-situ-measured
    PSDs (black) together with randomly drawn samples of measured PSDs
    (light grey) for a given altitude bin of a height of one kilometer.
    Colored lines on top show the corresponding mean retrieved PSD for
    different assumed particle shapes.}
  \label{fig:in_situ_psds_radar_only}
\end{figure}

As these results suggest, combining radar with passive microwave observations
may help to better constrain the PSD of ice hydrometeors at the base of clouds.
Since for air- and space-borne observations only microwave observations can
sense the base of thick clouds, this points towards a unique synergy between
these types of observations.


\section{Conclusions}
\label{sec:conclusions}

The main result from the experiments presented in this study is that we were
able to find two ice particle shapes for which the results of the combined
retrieval were consistent both with the observations as well as the
in-situ-measured IWC and PSDs. Considering the co-location issues that affected
the other two flights, we interpret this as a cautious indication of the
validity of the retrieval implementation. Since the ARTS radiative transfer
model and optical properties from the ARTS single-scattering database constitute
a crucial component of the retrieval, the results indirectly also validate
those. Although the evidence is limited, we therefore conclude that both
retrieval as well as the radiative transfer modeling using ARTS and the ARTS
single-scattering database work reliably across the millimeter- and
sub-millimeter domain given that ice hydrometeors are suitably represented
in the retrieval.

Furthermore, we found evidence that a synergistic retrieval that uses active and
passive microwave observations can help to better characterize the PSD of ice
hydrometeors at the base of clouds. This indicates that such retrievals can be
used to study the microphysical properties of clouds and thus help to improve
their representation in weather and climate models. However, also here a
suitable representation of the ice particle shape in the simulations is required
to ensure reliable results.

At the same time, we found no evidence of a signal that could help to constrain
the ice particle shape based solely on remote sensing observations. This means
that more work is needed to better constrain the shape a priori or by adding
further increasing the amount of observations used in the retrieval. For the
MARSS and ISMAR sensors, which are scanning radiometers, it would be possible to
incorporate additional viewing angles into the retrieval. Making such retrieval
computationally tractable, however, would require extensive development of the
radiative transfer model used in the retrieval.

Although further work will be required, this study demonstrates the feasibility
and potential of synergistic retrievals of ice hydrometeors combining active
and passive observations at millimeter and sub-millimeter wavelengths. With
regards to the upcoming ICI sensor, the retrieval may be used to further study
the representation of ice hydrometeors in different cloud systems, which will be
required to use these novel observations in retrievals and data assimilation.
Since the combined retrieval can better constrain the PSD of ice hydrometeors,
another potential application is to study cloud formation and representation in
models.



%% The following commands are for the statements about the availability of data sets and/or software code corresponding to the manuscript.
%% It is strongly recommended to make use of these sections in case data sets and/or software code have been part of your research the article is based on.

\codeavailability{All code used to produce the results in this study is available through public repositories \citep{mcrf}.} %% use this section when having only software code available


\dataavailability{} %% use this section when having only data sets available



\appendix



\noappendix       %% use this to mark the end of the appendix section

%% Regarding figures and tables in appendices, the following two options are possible depending on your general handling of figures and tables in the manuscript environment:

%% Option 1: If you sorted all figures and tables into the sections of the text, please also sort the appendix figures and appendix tables into the respective appendix sections.
%% They will be correctly named automatically.

%% Option 2: If you put all figures after the reference list, please insert appendix tables and figures after the normal tables and figures.
%% To rename them correctly to A1, A2, etc., please add the following commands in front of them:

\appendixfigures  %% needs to be added in front of appendix figures

\appendixtables   %% needs to be added in front of appendix tables

%% Please add \clearpage between each table and/or figure. Further guidelines on figures and tables can be found below.



\authorcontribution{Simon Pfreundschuh has performed the retrieval calculations
  and data analysis as well as written the manuscript. Patrick Eriksson, Stefan
  A. Buehler, Patrick Eriksson, Manfred Brath, David Duncan and Simon
  Pfreundschuh have collaborated on the study that lead to the development of
  the presented algorithm. Stuart Fox, Florian Ewald and Julien Delanoë have
  provided the flight campaign data, guidance regarding their usage and
  contributed to the interpretation and discussion of the retrieval results.}

\competinginterests{No competing interests are present} %% this section is mandatory even if you declare that no competing interests are present

\begin{acknowledgements}
The combined and radar-only was developed as part of the ESA-funded study
``Scientific Concept Study for Wide-Swath High-Resolution Cloud Profiling''
(Contract number: 4000119850/17/NL/LvH). The authors would like to thank
study manager Tobias Wehr for his valuable input and guidance during the study.

The work of SP, PE and RE on this study was financially supported by the Swedish National Space Agency
(SNSA) under grants 150/14 and 166/18.

SB was supported by the Deutsche Forschungsgemeinschaft (DFG, German Research
Foundation) under Germany's Excellence Strategy --- EXC 2037 'Climate, Climatic
Change, and Society' --- Project Number: 390683824, contributing to the Center
for Earth System Research and Sustainability (CEN) of Universit\"{a}t Hamburg.

The computations for this study were performed using several freely available programming
languages and software packages, most prominently the Python language
\citep{python}, the IPython computing environment \citep{ipython}, the numpy
package for numerical computing \citep{numpy} and matplotlib for generating
figures \citep{matplotlib}.

The computations were performed on resources at Chalmers Centre for
Computational Science and Engineering (C3SE) provided by the Swedish National
Infrastructure for Computing (SNIC).
\end{acknowledgements}




%% REFERENCES

%% The reference list is compiled as follows:


%% Since the Copernicus LaTeX package includes the BibTeX style file copernicus.bst,
%% authors experienced with BibTeX only have to include the following two lines:

\bibliographystyle{copernicus}
\bibliography{references}
%%
%% URLs and DOIs can be entered in your BibTeX file as:
%%
%% URL = {http://www.xyz.org/~jones/idx_g.htm}
%% DOI = {10.5194/xyz}


%% LITERATURE CITATIONS
%%
%% command                        & example result
%% \citet{jones90}|               & Jones et al. (1990)
%% \citep{jones90}|               & (Jones et al., 1990)
%% \citep{jones90,jones93}|       & (Jones et al., 1990, 1993)
%% \citep[p.~32]{jones90}|        & (Jones et al., 1990, p.~32)
%% \citep[e.g.,][]{jones90}|      & (e.g., Jones et al., 1990)
%% \citep[e.g.,][p.~32]{jones90}| & (e.g., Jones et al., 1990, p.~32)
%% \citeauthor{jones90}|          & Jones et al.
%% \citeyear{jones90}|            & 1990



%% FIGURES

%% When figures and tables are placed at the end of the MS (article in one-column style), please add \clearpage
%% between bibliography and first table and/or figure as well as between each table and/or figure.


%% ONE-COLUMN FIGURES

%%f
%\begin{figure}[t]
%\includegraphics[width=8.3cm]{FILE NAME}
%\caption{TEXT}
%\end{figure}
%
%%% TWO-COLUMN FIGURES
%
%%f
%\begin{figure*}[t]
%\includegraphics[width=12cm]{FILE NAME}
%\caption{TEXT}
%\end{figure*}
%
%
%%% TABLES
%%%
%%% The different columns must be seperated with a & command and should
%%% end with \\ to identify the column brake.
%
%%% ONE-COLUMN TABLE
%
%%t
%\begin{table}[t]
%\caption{TEXT}
%\begin{tabular}{column = lcr}
%\tophline
%
%\middlehline
%
%\bottomhline
%\end{tabular}
%\belowtable{} % Table Footnotes
%\end{table}
%
%%% TWO-COLUMN TABLE
%
%%t
%\begin{table*}[t]
%\caption{TEXT}
%\begin{tabular}{column = lcr}
%\tophline
%
%\middlehline
%
%\bottomhline
%\end{tabular}
%\belowtable{} % Table Footnotes
%\end{table*}
%
%%% LANDSCAPE TABLE
%
%%t
%\begin{sidewaystable*}[t]
%\caption{TEXT}
%\begin{tabular}{column = lcr}
%\tophline
%
%\middlehline
%
%\bottomhline
%\end{tabular}
%\belowtable{} % Table Footnotes
%\end{sidewaystable*}
%
%
%%% MATHEMATICAL EXPRESSIONS
%
%%% All papers typeset by Copernicus Publications follow the math typesetting regulations
%%% given by the IUPAC Green Book (IUPAC: Quantities, Units and Symbols in Physical Chemistry,
%%% 2nd Edn., Blackwell Science, available at: http://old.iupac.org/publications/books/gbook/green_book_2ed.pdf, 1993).
%%%
%%% Physical quantities/variables are typeset in italic font (t for time, T for Temperature)
%%% Indices which are not defined are typeset in italic font (x, y, z, a, b, c)
%%% Items/objects which are defined are typeset in roman font (Car A, Car B)
%%% Descriptions/specifications which are defined by itself are typeset in roman font (abs, rel, ref, tot, net, ice)
%%% Abbreviations from 2 letters are typeset in roman font (RH, LAI)
%%% Vectors are identified in bold italic font using \vec{x}
%%% Matrices are identified in bold roman font
%%% Multiplication signs are typeset using the LaTeX commands \times (for vector products, grids, and exponential notations) or \cdot
%%% The character * should not be applied as mutliplication sign
%
%
%%% EQUATIONS
%
%%% Single-row equation
%
%\begin{equation}
%
%\end{equation}
%
%%% Multiline equation
%
%\begin{align}
%& 3 + 5 = 8\\
%& 3 + 5 = 8\\
%& 3 + 5 = 8
%\end{align}
%
%
%%% MATRICES
%
%\begin{matrix}
%x & y & z\\
%x & y & z\\
%x & y & z\\
%\end{matrix}
%
%
%%% ALGORITHM
%
%\begin{algorithm}
%\caption{...}
%\label{a1}
%\begin{algorithmic}
%...
%\end{algorithmic}
%\end{algorithm}
%
%
%%% CHEMICAL FORMULAS AND REACTIONS
%
%%% For formulas embedded in the text, please use \chem{}
%
%%% The reaction environment creates labels including the letter R, i.e. (R1), (R2), etc.
%
%\begin{reaction}
%%% \rightarrow should be used for normal (one-way) chemical reactions
%%% \rightleftharpoons should be used for equilibria
%%% \leftrightarrow should be used for resonance structures
%\end{reaction}
%
%
%%% PHYSICAL UNITS
%%%
%%% Please use \unit{} and apply the exponential notation


\end{document}
